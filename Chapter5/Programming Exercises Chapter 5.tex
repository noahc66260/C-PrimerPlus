\documentclass[10 pt]{amsart}

\usepackage{amssymb,latexsym}
\usepackage{graphicx}

% For the cpart environment, although it would probably be better in the
% future to implement this with a list environment.
\newlength{\cgap}
\settowidth{\cgap}{\qquad \textbf{x. }}
\newlength{\cwidth}
\setlength{\cwidth}{\textwidth}
\addtolength{\cwidth}{-\cgap}
\newenvironment{cpart}[2][\cwidth]
	{\\ \phantom{\qquad}\textbf{#2. }\begin{minipage}[t]{#1}}
	{\end{minipage}}


\newcommand{\ttt}[1]{\texttt{#1}}
\newcommand{\ttb}[1]{\pmb{\texttt{#1}}}
% Macros, all must be filled out
\newcommand{\ChapNum}{5}


\begin{document}

	\title
	[Chapter \ChapNum]
	{C++ Primer Plus, 5$^\text{th}$ Edition \\
	Programming Exercises \\
	Chapter \ChapNum}

	\maketitle

	\begin{cpart}{1}
		Write a program that requests the user to enter two integers.
		The program should then calculate and report the sum of all
		the integers between and including the two integers.
		At this point, assume that the smaller integer is entered first.
		For example, if the user enters \ttt{2} and \ttt{9},
		the program should report that the sum of all the integers
		from 2 to 9 is 44.
	\end{cpart}
	\vspace{2ex}

	\begin{cpart}{2}
		Write a program that asks teh user to type in numbers.
		After each entry, the program should report the cumulative sum
		of teh entries to date.
		The program should terminate when the user enters \ttb{0}.
	\end{cpart}
	\vspace{2ex}

	\begin{cpart}{3}
		Daphne invests \$100 at 10\% simple interest.
		That is, every year, the investment earns 10\% of the original
		investment, or \$10 each and every year: \vspace{2ex} \\
		interest = 0.10 $\times$ original balance. \vspace{2ex} \\
		At the same time, Cleo invests \$100 at 5\% compound interest.
		That is, interest is 5\% of the current balance, including
		previous additions of interest: \vspace{2ex} \\
		interest = 0.05 $\times$ current balance \vspace{2ex} \\
		Cleo earns 5\% of \$100 the first year, giving her \$105. 
		The next year she earns 5\% of \$105, or \$5.25, and so on.
		Write a program that finds how many years it takes for the value
		of Cleo's investment to exceed the value of Daphne's investment
		and then displays the value of both investments at the same 
		time.
	\end{cpart}
	\vspace{2ex}

	\begin{cpart}{4}
		You sell the Book {\bf \emph{C++ for Fools}}. 
		Write a program that has you enter a year's worth of monthly
		sales (in terms of number of books, not of money).
		The program should use a loop to prompt you by month, using an 
		array of \verb+char *+ (or an array of \ttt{string} objects,
		if you prefer) initialized to the month strings and storing
		the input data in an array of \ttt{int}.
		Then, the program should find the sum of the array contents and
		report the total sales for the year.
	\end{cpart}
	\vspace{2ex}

	\begin{cpart}{5}
		Do Programming Exercise 4, but use a two-dimensional array to 
		store input for 3 years of monthly sales.
		Report the total sales for each individual year and for
		the combined years.
	\end{cpart}
	\vspace{2ex}

	\begin{cpart}{6}
		Design a structure called \ttt{car} that holds the following
		information about an automobile:
		its make, as a string in a character array or in a \ttt{string}
		object, and the year it ws build, as an integer.
		Write a program that asks the user how many cars to catalog.
		The program should then use \ttt{new} to create a dynamic array
		of that many \ttt{car} structures.
		Next, it should prompt the user to input the make (which 
		might consist of more than one word) and year information for
		each structure.
		Note that this requires some care because it alternates reading
		strings with numeric data (see Chapter 4).
		Finally, it should display the contents of each structure.
		A sample run should look something like the following:
		{\ttfamily
			\begin{tabbing}
				\phantom{\qquad}\=\phantom{\qquad}\=\phantom{\qquad}\= \\
				How many cars do you wish to catalog?\ttb{ 2} \\
				Car \# 1: \\
				Please enter the make:\ttb{ Hudson Hornet} \\
				Please enter the year made:\ttb{ 1952} \\
				Car \# 2: \\
				Please enter the make:\ttb{ Kaiser} \\
				Please enter the year made:\ttb{ 1951} \\
				Here is your collection: \\
				1952 Hudson HOrnet \\
				1951 Kaiser
			\end{tabbing}
		}
	\end{cpart}
	\vspace{2ex}

	\begin{cpart}{7}
		Write a program that uses an array of \ttt{char} and a loop 
		to read one word at a time until the word \ttt{done} is entered.
		The program should then report the number of words entered
		(not counting \ttt{done}). 
		A sample run could look like this:
		{\ttfamily
			\begin{tabbing}
				\phantom{\qquad}\=\phantom{\qquad}\=\phantom{\qquad}\= \\
				Enter words (to stop, type the word done): \\
				\ttb{anteater birthday category dumpster}\\
				\ttb{envy finagle geometry done for sure}\\
				You entered a total of 7 words.
			\end{tabbing}
		}\vspace{2ex} \\
		You should include the \ttt{cstring} header file and use the
		\ttt{stringcmp()} function to make the comparison test. 
	\end{cpart}
	\vspace{2ex}

	\begin{cpart}{8}
		Write a program that matches the description in Programming
		Exercise 7, but use a \ttt{string} class object instead of
		an array.
		Include the \ttt{string} header file and use a relational
		operator to make the comparison test.
	\end{cpart}
	\vspace{2ex}

	\begin{cpart}{9}
		Write a program using nested loops that asks the user to enter
		a value for the number of rows to display.
		It should then display that many rows of asterisks,
		with one asterisk in the first row, two in the second row,
		and so on.
		For each row, the asterisks are preceded by the number of
		periods needed to make all the rows display the total number
		of characters equal to the number of rows.
		A sample run would look like this:
		{\ttfamily
			\begin{tabbing}
				\phantom{\qquad}\=\phantom{\qquad}\=\phantom{\qquad}\= \\
				Enter number of rows:\ttb{ 5}\\
				....* \\
				...** \\
				..*** \\
				.**** \\
				*****
			\end{tabbing}
		}
	\end{cpart}

\end{document}

regarding tabbing environments:
\= (set tab)
\> (advance to next tab stop)
\<
\+ (indent; move margin right)
\- (unindent; move margin left)
\'
\`
\\ (end of line; newline)
\kill (ignore preceding text; use only for spacing)



{\ttfamily
	\begin{tabbing}
		\phantom{\qquad}\=\phantom{\qquad}\=\phantom{\qquad}\= \\
		
	\end{tabbing}
}











