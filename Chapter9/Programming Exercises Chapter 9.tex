\documentclass[10 pt]{amsart}

\usepackage{amssymb,latexsym}
\usepackage{graphicx}

% For the cpart environment, although it would probably be better in the
% future to implement this with a list environment.
\newlength{\cgap}
\settowidth{\cgap}{\qquad \textbf{x. }}
\newlength{\cwidth}
\setlength{\cwidth}{\textwidth}
\addtolength{\cwidth}{-\cgap}
\newenvironment{cpart}[2][\cwidth]
	{\\ \noindent\phantom{\qquad}\textbf{#2. }\begin{minipage}[t]{#1}}
	{\end{minipage}}
\newenvironment{cpartContinued}[2][\cwidth]
	{\\ \noindent\phantom{\qquad}%
		\phantom{\textbf{#2. }}\begin{minipage}[t]{#1}}
	{\end{minipage}}

\newcommand{\ttt}[1]{\texttt{#1}}
\newcommand{\ttb}[1]{\pmb{\texttt{#1}}}
% Macros, all must be filled out
\newcommand{\ChapNum}{9}

\begin{document}

	\title
	[Chapter \ChapNum]
	{C++ Primer Plus, 5$^\text{th}$ Edition \\
	Programming Exercises \\
	Chapter \ChapNum}

	\maketitle

	\begin{cpart}{1}
		Here is a header file:
		{\ttfamily
			\begin{tabbing}
				\phantom{\qquad}\=\phantom{\qquad}\=\phantom{\qquad}\= \\
				// golf.h -- for pe9-1.cpp \\
				\\
				const int Len = 40; \\
				struct golf \\
				\{ \\
				\> 	char fullname[Len]; \\
				\> 	int handicap; \\
				\}; \\
				\\
				// non-interactive version: \\
				// function sets golf structure to provided name, handicap \\
				// using values passed as arguments to the function \\
				void setgolf(golf & g, const char * name, int hc); \\
				\\
				// interactive version: \\
				// function solicits name and handicap from user \\
				// and sets the members of g to teh values entered \\
				// returns 1 if name is entered, 0 if name is empty string \\
				int setgolf(golf & g); \\
				\\
				// function resets handicap to new value \\
				void handicap(golf & g, int hc); \\
				\\
				// function displays contents of golf structure \\
				void showgolf(const golf & g);
			\end{tabbing}
		}
		\phantom{\quad} \\
		Note that \ttt{setgolf()} is overloaded. 
		Using the first version of \ttt{setgolf()} would look like this:
		\vspace{2ex} \\
		\ttt{golf ann;} \\
		\ttt{setgolf(ann, "Ann Birdfree", 24);}
		%\vspace{2ex} \\
	\end{cpart}
	
	\begin{cpartContinued}{1}
		The function call provides the information that's stored in the 
		\ttt{ann} structure.
		Using the second version of \ttt{setgolf()} would look like
		this: \\[2ex]
		{\ttfamily
			golf andy; \\
			setgolf(andy);
		} \\[2ex]
		The function would prompt the user to enter the name and handicap
		and store them in the \ttt{andy} structure.
		This function could (but doesn't need to) use the first
		version internally. \\[2ex]
		Put together a multifile program based on this header.
		One file, named \ttt{golf.cpp}, should provide suitable
		function definitions to match the prototypes
		in the header file.
		A second file should contain \ttt{main()} and demonstrate all 
		the features of the prototyped functions.
		For example, a loop should solicit input for an array of golf
		structures and terminate when teh array is full or the
		user enters an empty string for the golfer's name.
		The \ttt{main()} function should use only the prototyped 
		functions to access the golf structures.
	\end{cpartContinued} 
	\\[2ex]

	\begin{cpart}{2}
		Redo Listing 9.8, replacing the character array with a
		\ttt{string} object.
		The program should no longer have to check wheter the input
		string fits, and it can compare the input string to "" to 
		check for an empty line.
	\end{cpart}
	\vspace{2ex}

	\begin{cpart}{3}
		Begin with the following structure declaration:
		{\ttfamily
			\begin{tabbing}
				\phantom{\qquad}\=\phantom{\qquad}\=\phantom{\qquad}\= \\
				struct chaff \\
				\{ \\
				\> 	char dross[20]; \\
				\> 	int slag; \\
				\};
			\end{tabbing}
		}
		\phantom{\quad} \\
		Write a program that uses placement \ttt{new} to place
		an array of two such structures in a buffer.
		Then assign values to the structure members (remembering to
		use \ttt{strcpy()} for the \ttt{char} array) and use
		a loop to display the contents.
		Option 1 is to use a static array, like that in Listing 9.9,
		for the buffer.
		Option 2 is to use regular \ttt{new} to allocate the buffer.
	\end{cpart}
	\vfill
	\newpage

	\begin{cpart}{4}
		Write a three-file program based on the following namespace:
		{\ttfamily
			\begin{tabbing}
				\phantom{\qquad}\=\phantom{\qquad}\=\phantom{\qquad}\= \\
					namespace SALES \\
					\{ \\
					\> 	const int QUARTERS = 4; \\
					\> 	struct Sales \\
					\> 	\{
					\> \>		double sales[QUARTERS]; \\
					\> \>		double average; \\
					\> \>		double max; \\
					\> \>		double min; \\
					\> 	\}; \\
					\\
					\> 	// copies the lessoer of 4 or n items from
								the array ar \\
					\> 	// to the sales member of s and computes 
								and stores the \\
					\> 	// average, maximum, and minimum values of the
								entered items; \\
					\> 	// remaining elements of sales, if any, set to 0 \\
					\> 	void setSales(Sales & s, const double ar[], int n); \\
					\\
					\> 	// gathers sales for 4 quarters interactively,
								stores them \\
					\> 	// in the sales member of s and computes and
								stores the \\
					\> 	// average, maximum, and minimum values \\
					\> 	void setSales(Sales & s); \\
					\\
					\> 	// display all information in structure s \\
					\> 	void showSales(const Sales & s); \\
					\}
			\end{tabbing}
		}
		\phantom{\quad} \\
		The first file should be a header file that contains the 
		namespace.
		The second file should be a source code file that extends the
		namespace to provide definitions for the three prototyped 
		functions.
		The third file should declare two \ttt{Sales} objects.
		It should use the interactive version of \ttt{setSales()}
		to provide values for one structure and the non-interactive
		version of \ttt{setSales()} to provide values for the second
		structure.
		It should display the contents of both structures by using
		\ttt{showSales()}.
	\end{cpart}
	\vspace{2ex}

\end{document}

regarding tabbing environments:
\= (set tab)
\> (advance to next tab stop)
\<
\+ (indent; move margin right)
\- (unindent; move margin left)
\'
\`
\\ (end of line; newline)
\kill (ignore preceding text; use only for spacing)



{\ttfamily
	\begin{tabbing}
		\phantom{\qquad}\=\phantom{\qquad}\=\phantom{\qquad}\= \\
		
	\end{tabbing}
}











