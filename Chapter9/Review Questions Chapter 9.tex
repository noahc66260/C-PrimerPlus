\documentclass{amsart}

\usepackage{amssymb,latexsym, verbatim}
\thispagestyle{empty}
\pagestyle{empty}

\begin{document}
\begin{center}
	\Large {\bfseries
	\emph{C++ Primer Plus, $5^{\text{th}}$ Edition} by Stephen Prata \\
	Chapter 9: Memory Models and Namespaces \\
	Review Questions} \normalsize \vspace{5ex}
\end{center}

% Note: in order to really get the formatting I want, I need to create my own environment. It would be similar to the enumerate environment, but instead of enclosing the enumerator in parentheses, there would just be a period after it. For example, rather than (1), we would have 1. instead.

\noindent 1. 
\begin{minipage}[t]{11.5 cm}
	What storage scheme would you use for the following situations?
\end{minipage} \\[1ex]
\phantom{1. }a.
\begin{minipage}[t]{11.5 cm}
	\verb+homer+ is a formal argument (parameter) to a function. \\[1ex]
	{\slshape Automatic.} \\
	{} % space for comments to be included
\end{minipage} \\[1ex]
\phantom{1. }b.
\begin{minipage}[t]{11.5 cm}
	The \verb+secret+ variable is to be shared by two files. \\[1ex]
	{\slshape Global (static with external linkage).}\\
	{} % space for comments to be included
\end{minipage} \\[1ex]
\phantom{1. }c.
\begin{minipage}[t]{11.5 cm}
	The \verb+topsecret+ variable is to be shared by the functions in one file but hidden from other files. \\[1ex]
	{\slshape Static with internal linkage.} \\
	{} % space for comments to be included
\end{minipage} \\[1ex]
\phantom{1. }d.
\begin{minipage}[t]{11.5 cm}
	\verb+beencalled+ keeps track of how many times the function containing it has been called. \\[1ex]
	{\slshape Static with no linkage.} \\
	{} % space for comments to be included
\end{minipage}
\vfill

\noindent 2. 
\begin{minipage}[t]{11.5 cm}
	Describe the differences between a \verb+using+ declaration and a \verb+using+ directive.
\end{minipage} \\[1ex]
\phantom{2. } 
\begin{minipage}[t]{11.5 cm}
	{\slshape 
	The \verb+using+ declaration makes a single function, structure, 
	constant, or member of the namespace available throughout the
	scope of the declaration.
	The \verb+using+ directive makes every function, structure, 
	constant, and member of the namespace available throughout the
	scope of the declaration.
	} 
\end{minipage} 
\vfill

\noindent 3. 
\begin{minipage}[t]{11.5 cm}
	Rewrite the following so that it doesn't use \verb+using+ declarations or \verb+using+ directives:
	\begin{verbatim}
		#include<iostream>
		using namespace std;
		int main()
		{
		    double x;
		    cout << "Enter value: ";
		    while (! (cin >> x))
		    {
		        cout << "Bad input. Please enter a number: ";
		        cin.clear();
		        while (cin.get() != '\n')
		            continue;
		    }
		    cout << "Value = " << x << endl;
		    return 0;
		}
	\end{verbatim}
\end{minipage}
\newpage

\phantom{}
\vfill
\phantom{3. } 
\begin{minipage}[t]{11.5 cm}
	{\slshape See the following code:}
	\begin{verbatim}
		#include<iostream>
		int main()
		{
		    double x;
		    std::cout << "Enter value: ";
		    while (! (std::cin >> x))
		    {
		        std::cout << "Bad input. Please enter a number: ";
		        std::cin.clear();
		        while (std::cin.get() != '\n')
		            continue;
		    }
		    std::cout << "Value = " << x << std::endl;
		    return 0;
		}
	\end{verbatim} 
\end{minipage} 
\vfill

\noindent 4. 
\begin{minipage}[t]{11.5 cm}
	Rewrite the following so that it uses \verb+using+ declarations instead of the \verb+using+ directive:
	\begin{verbatim}
		#include<iostream>
		using namespace std;
		int main()
		{
		    double x;
		    cout << "Enter value: ";
		    while (! (cin >> x))
		    {
		        cout << "Bad input. Please enter a number: ";
		        cin.clear();
		        while (cin.get() != '\n')
		            continue;
		    }
		    cout << "Value = " << x << endl;
		    return 0;
		}
	\end{verbatim}
\end{minipage}
\vfill
\newpage

\phantom{}
\vfill
\phantom{2. } 
\begin{minipage}[t]{11.5 cm}
	{\slshape See the following code:}
	\begin{verbatim}
		#include<iostream>
		int main()
		{
		    using std::cout;
		    using std::cin;
		    using std::endl;
		    double x;
		    cout << "Enter value: ";
		    while (! (cin >> x))
		    {
		        cout << "Bad input. Please enter a number: ";
		        cin.clear();
		        while (cin.get() != '\n')
		            continue;
		    }
		    cout << "Value = " << x << endl;
		    return 0;
		}
	\end{verbatim} 
\end{minipage} 
\vfill

\noindent 5. 
\begin{minipage}[t]{11.5 cm}
	Say that the \verb+average(3,6)+ function returns an \verb+int+ average of the two \verb+int+ arguments when it is called in one file, and it returns a \verb+double+ average of the two \verb+int+ arguments when it is called in a second file in the same program. How could you set this up?
\end{minipage} \\[1ex]
\phantom{3. } 
\begin{minipage}[t]{11.5 cm}
	{\slshape 
	We could create a function called \verb+average()+ in a given
	namespace which accepts two \verb+int+s and returns an \verb+double+.
	Then we could create a program were we declare and define the 
	function \verb+average()+ which accepts two \verb+int+s and returns
	a \verb+double+.
	If we wanted to use the function that returned an \verb+int+, we 
	would call \verb+average()+. 
	If we wanted to use the function that returned a double, we would
	use \verb+<namespace>::average()+ where \verb+<namespace>+
	represents the namespace we created for the function.
	} 
\end{minipage} 
\vfill
\newpage

\phantom{}
\vfill
\noindent 6. 
\begin{minipage}[t]{11.5 cm}
	What will the following two-file program display?
	\begin{verbatim}
		// file1.cpp
		#include<iostream>
		using namespace std;
		void other();
		void another();
		int x = 10;
		int y;

		int main()
		{
		    cout << x << endl;
		    {
		        int x = 4;
		        cout << x << endl;
		        cout << y << endl;
		    }
		    other();
		    another();
		    return 0;
		}

		void other()
		{
		    int y = 1;
		    cout << "Other: " << x << ", " << y << endl;
		}

		// file2.cpp
		#include<iostream>
		using namespace std;
		extern int x;
		namespace
		{
		    int y = -4;
		}

		void another()
		{
		    cout << "another (): " << x << ", " << y << endl;
		}
	\end{verbatim}
\end{minipage}
\phantom{3. } 
\begin{minipage}[t]{11.5 cm}
	{\slshape It would print the following:} \\
	\phantom{\qquad}10 \\
	\phantom{\qquad}4 \\
	\phantom{\qquad}(garbage) \\
	\phantom{\qquad}Other: 10, 1 \\
	\phantom{\qquad}another (): 10, -4 \\
%
%	Note: (garbage) could be anything since \verb+y+ is not defined
%	before its first use. 
\end{minipage} 
\vfill
\newpage

\phantom{}
\vfill
\noindent 7. 
\begin{minipage}[t]{11.5 cm}
	What will the following program display?
	\begin{verbatim}
		#include <iostream>
		using namespace std;
		void other();
		namespace n1
		{
		    int x = 1;
		}

		namespace n2
		{
		    int x = 2;
		}


		int main()
		{
		    using namespace n1;
		    cout << x << endl;
		    {
		        int x = 4;
		        cout << x << ", " << n1::x << ", " << n2::x << endl;
		    }
		    using n2::x;
		    cout << x << endl;

		    other();
		    return 0;
		}

		void other()
		{
		    using namespace n2;
		    cout << x << endl;
		    {
		        int x = 4;
		        cout << x << ", " << n1::x << ", " << n2::x << endl;
		    }
		    using n2::x;
		    cout << x << endl;
		}
	\end{verbatim}
\end{minipage}
\phantom{2. } 
\begin{minipage}[t]{11.5 cm}
	{\slshape We would see the following:}\\
	\phantom{\qquad}1 \\
	\phantom{\qquad}4, 1, 2 \\
	\phantom{\qquad}2 \\
	\phantom{\qquad}2 \\
	\phantom{\qquad}4, 1, 2 \\
	\phantom{\qquad}2
\end{minipage} 
\vfill

\end{document}

Here is the format for questions that include multiple parts:

\noindent X. 
\begin{minipage}[t]{11.5 cm}
	The question
\end{minipage} \\[1ex]
\phantom{1. }a.
\begin{minipage}[t]{11.5 cm}
	part a \\[1ex]
	{\slshape The answer. You can also type code: \verb+cout << "Hello" << endl;+} \\
	{} % space for comments to be included
\end{minipage} \\[1ex]
\phantom{1. }b.
\begin{minipage}[t]{11.5 cm}
	part b \\[1ex]
	{\slshape The answer}\\
	{} % space for comments to be included
\end{minipage} \\[1ex]
\phantom{1. }c.
\begin{minipage}[t]{11.5 cm}
	part c \\[1ex]
	{\slshape The answer} \\
	{} % space for comments to be included
\end{minipage} \\[1ex]
\phantom{1. }d.
\begin{minipage}[t]{11.5 cm}
	part d \\[1ex]
	{\slshape The answer} \\
	{} % space for comments to be included
\end{minipage}
\vfill

\noindent X. 
\begin{minipage}[t]{11.5 cm}
	The question
\end{minipage} \\[1ex]
\phantom{3. } 
\begin{minipage}[t]{11.5 cm}
	{\slshape The answer.} 
\end{minipage} 
\vfill
