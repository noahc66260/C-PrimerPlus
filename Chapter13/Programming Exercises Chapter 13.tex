\documentclass[10 pt]{amsart}

\usepackage{amssymb,latexsym}
\usepackage{graphicx, setspace}
% the setspace package allows us use of the 
% spacing environment (\begin{spacing}{second arg}) where
% second arg is a number to multiply to the spacing factor.
% use 2 for double space, 1 for single space, etc.

% For the cpart environment, although it would probably be better in the
% future to implement this with a list environment.
\newlength{\cgap}
\settowidth{\cgap}{\textbf{x. }}
\newlength{\cwidth}
\setlength{\cwidth}{\textwidth}
\addtolength{\cwidth}{-\cgap}
\newenvironment{cpart}[2][\cwidth]
	{%		
		\\ %
		\textbf{#2. }%
		\begin{minipage}[t]{#1}%
		\setlength{\parindent}{0pt}%
		\setlength{\parskip}{2ex}%
	}
	{%
		\end{minipage}%
	}
\newenvironment{cpartContinued}[2][\cwidth]
	{%		
		\\ %
		\textbf{#2. (continued)}%
		\\
		\phantom{#2. }
		\begin{minipage}[t]{#1}%
		\setlength{\parindent}{0pt}%
		\setlength{\parskip}{2ex}%
	}
	{%
		\end{minipage}%
	}


% set paragraph spacing like that in the book
\setlength{\parindent}{0pt}
\setlength{\parskip}{2ex}

% these new commands will make typing different formats easier.
\newcommand{\ttt}[1]{\texttt{#1}}
\newcommand{\ttb}[1]{\pmb{\texttt{#1}}}
\newcommand{\tbs}{\textbackslash}
% Macros, all must be filled out
\newcommand{\ChapNum}{13}

\begin{document}
	\title
	[Chapter \ChapNum]
	{C++ Primer Plus, 5$^\text{th}$ Edition \\
	Programming Exercises \\
	Chapter \ChapNum}

	\maketitle

	\begin{cpart}{1}
		Start with the following class declaration:
		{\ttfamily
			\begin{tabbing}
				\phantom{\qquad}\=\hspace{4 cm}\= \\
				// base class \\
				class Cd \{  // represents a CD stack \\
				private: 
				\+ \\
				char performers[50]; \\
				char label[20]; \\
				int selections; \> // number of selections \\
				double playtime; \> // playing time in minutes \\
				\< public: \\
				Cd(char * s1, char * s2, int n, double x); \\
				Cd(const Cd & d); \\
				Cd(); \\
				$\sim$Cd(); \\
				void Report() const; \quad // reports all CD data \\
				Cd \& operator=(const Cd \& cd); \\
				\< \};
			\end{tabbing}
		}
		Derive a \ttt{Classic} class that adds an array of \ttt{char}
		members that will hold a string idenfitying the primary work
		on the CD.
		If the base class requires that any function sbe virtual, modify
		the base-class declaration to make it so.
		If a declared method is not needed, remove it from the
		defintion. 
		Test your product with the following program:
	\end{cpart}
	\newpage
	\begin{cpartContinued}{1}
		{\ttfamily
			\begin{tabbing}
				\phantom{\qquad}\=\hspace{4cm}\= \\
				\#include <iostream> \\
				using namespace std; \\
				\#include "classic.h" \> \>
					// which will contain \#include "cd.h" \\
				void Bravo(const Cd \& disk); \\
				\{
				\+ \\
					Cd c1("Beatles", "Capitol", 14, 35.5); \\
					Classic c2 = 
						Classic("Piano Sonata in B flat, Fantasia in C", 
							"Alfred Brendel", "philips", 2, 57.17); \\
					Cd * pcd = \&c1; \\
					\\
					cout << "Using object directly:\tbs n"; \\
					c1.Report(); \> // use Cd method \\
					c2.Report(); \> // use Classic method \\
					\\
					cout << "Using type cd * pointer to objects:\tbs n"; \\
					pcd->Report(); \> // use Cd method for cd object \\
					pcd = \&c2; \\
					pcd->Report(); \> // use Classic method for 
						classic object \\
					\\
					cout << "Calling a function with a Cd reference 
								argument:\tbs n"; \\
					Bravo(c1); \\
					Bravo(c2); \\
					\\
					cout << "Testing assignment: "; \\
					Classic copy; \\
					copy = c2; \\
					copy.Report(); \\
					\\
					return 0; \\
				\< \}; \\
				\- \\
				void Bravo(const Cd \& disk) \\
				\{ \\
				\> disk.Report(); \\
				\};
			\end{tabbing}
		}
	\end{cpartContinued}

	\begin{cpart}{2}
		Do Programming Exercise 1, but use dynamic memory allocation
		instead of fixed-size arrays for the various strings
		tracked by the two classes.
	\end{cpart}

	\begin{cpart}{3}
		Revise the \ttt{BaseDMA-lacksDMA-hasDMA} class hierarchy so that
		all three classes are derived from an ABC.
		Test the result with a program similar to the one in 
		Listing 13.10.
		That is, it should feature an array of pointers to the ABC
		and allow the user to make runtime decisions as to what
		types of objects are created.
		Add virtual \ttt{View()} methods to the class definitions
		to handle displaying data.
	\end{cpart}

	\begin{cpart}{4}
		The Benevolent Order of Programmers maintains a collection of 
		bottled port.
		To describe it, the BOP Portmaster has devised a \ttt{Port}
		class, as described here:
		{\ttfamily
			\begin{tabbing}
				\phantom{\qquad}\=\hspace{6cm}\=\\
				\#include <iostream> \\
				using namespace std; \\
				class Port \\
				\{ \\
				private:
				\+ \\
					char * brand; \\
					char style[20]; // i.e., tawny, ruby, vintage \\
					int bottles; \\
				\< public: \\
					Port(const char * br = "none", 
						const char * st = "none", int b = 0); \\
						Port(const Port \& p); \> // copy constructor \\
						virtual $\sim$Port() \{ delete [] brand; \} \\
						Port \& operator=(const Port \& p); \\
						Port \& operator+=(int b); \> // adds b to bottles \\
						Port \& operator-=(int b); \> 
							// substracts b from bottles, if available \\
						int BottleCount() const \{ return bottles; \} \\
						virtual void Show() const; \\
						friend ostream \& operator<<(ostream & os, 
																const Port \& p); \\
				\< \};
			\end{tabbing}
		}
		The \ttt{Show()} method presents information in the following
		format:
		
		{ \ttfamily
			Brand: Gallo \\
			Kind: tawny \\
			Bottles: 20 \\
		}

		The \ttt{operator<<()} function presents information in the
		following format (with no newline character at the end):

		\ttt{Gallo, tawny, 20}
	\end{cpart}
	\newpage

	\begin{cpartContinued}{4}
		The Portmaster completed the method definitions for the
		\ttt{Port} class and then derived the \ttt{VintagePort} class
		as follows before being relieved of his position for
		accidentally routing a bottle of '45 Cockburn to someone
		preparing an experimental barbecue sauce: 
		{\ttfamily
			\begin{tabbing}
				\phantom{\qquad}\=\hspace{6cm}\=\\	
				class VintagePort : public Port 
					// style necessarily = "vintage" \\
				\{ 
				\+ \\
				\< private: \\
					char * nickname; \> 
							// i.e., "The Noble" or "Old Velvet", etc. \\
					int year: \> // vintage year \\
				\< public: \\
					VintagePort(); \\
					VintagePort(const char * br, 
						int b, const char * nn, int y); \\
					VintagePort(const VintagePort \& vp); \\
					$\sim$VintagePort() \{ delete [] nickname; \} \\
					VintagePort \& operator=(const VintagePort \& vp); \\
					void Show() const; \\
					friend ostream \& 
						operator<<(ostream \& os, const VintagePort \& vp); \\
				\< \};
			\end{tabbing}
		}	
		You get a job of completing the \ttt{VintagePort} work.
		\begin{enumerate}%{a.}
			\item 
				Your first task is to re-create the \ttt{Port} method
				definitions because the former Portmaster immolated his
				upon being relieved. 
			\item
				Your second task is to explain why certain methods are
				redefined and others are not.
			\item
				Your third task is to explain why \ttt{operator=()}
				and \ttt{operator<<()} are not virtual.
			\item
				Your fourth task is to provide definitions for the 
				\ttt{VintagePort} methods.
		\end{enumerate}
	\end{cpartContinued}

\end{document}

regarding tabbing environments:
\= (set tab)
\> (advance to next tab stop)
\<
\+ (indent; move margin right)
\- (unindent; move margin left)
\'
\`
\\ (end of line; newline)
\kill (ignore preceding text; use only for spacing)

use \hspace{...} if you prefer

		{\ttfamily
			\begin{tabbing}
				\phantom{\qquad}\=\phantom{\qquad}\=\phantom{\qquad}\= \\
		
			\end{tabbing}
		}











