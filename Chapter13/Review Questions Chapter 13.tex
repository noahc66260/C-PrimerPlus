\documentclass{amsart}

\usepackage{amssymb,latexsym, verbatim}
\thispagestyle{empty}
\pagestyle{empty}

\begin{document}
\begin{center}
	\Large {\bfseries
	\emph{C++ Primer Plus, $5^{\text{th}}$ Edition} by Stephen Prata \\
	Chapter 13: Class Inheritance \\
	Review Questions} \normalsize \vspace{5ex}
\end{center}

% Note: in order to really get the formatting I want, I need to create my own environment. It would be similar to the enumerate environment, but instead of enclosing the enumerator in parentheses, there would just be a period after it. For example, rather than (1), we would have 1. instead.

\noindent 1. 
\begin{minipage}[t]{11.5 cm}
	What does a derived class inherit from a base class?
\end{minipage} \\[1ex]
\phantom{3. } 
\begin{minipage}[t]{11.5 cm}
	{\slshape 
	We inherit all private, public, and protected class member functions
	and variables. 
	However, the derived class does not inherit any constructors, 
	destructors, assignment operators, and friend functions of the
	base class
	} 
\end{minipage} 
\\[2ex]

\noindent 2. 
\begin{minipage}[t]{11.5 cm}
	What doesn't a derived class inherit from a base class?
\end{minipage} \\[1ex]
\phantom{2. } 
\begin{minipage}[t]{11.5 cm}
	{\slshape 
	See #1
	} 
\end{minipage} 
\\[2ex]

\noindent 3. 
\begin{minipage}[t]{11.5 cm}
	Suppose the return type for the \texttt{baseDMA::operator=()} function were defined as \texttt{void} instead of \texttt{baseDMA \&}. What effect, if any, would that have? What if the return type were \texttt{baseDMA} instead of \texttt{baseDMA \&}?
\end{minipage} \\[1ex]
\phantom{3. } 
\begin{minipage}[t]{11.5 cm}
	{\slshape 
	If the return type were void, then statements such as
	}
	\begin{align*}
		&\texttt{baseDMA b1;} \\
		&\texttt{baseDMA b2;} \\
		&\texttt{baseDMA b3;} \\
		&\texttt{b1 = b2 = b3;}
	\end{align*}
	would not be possible.
	Since the assignment operator operates from right to left,
	the last statement is setting the value of \verb+b1+ to 
	\verb+b2 = b3+, which has a return value of void. 
	If the return type was an object rather than a reference to 
	an object, then the copy constructor would be invoked to
	create the object returned. 
	This makes the program run a bit slower, but it is otherwise
	harmless unless \verb+baseDMA+ uses dynamic memory allocation
	and a copy constructor is not explicitly defined to do deep copying.
	However, the user must keep in mind that the object returned is
	not the object used to invoke the overloaded equality operator,
	so passing the returned object as a reference in a function which 
	modifies it will leave the original object unchanged. 
\end{minipage} 
\\[2ex]

\noindent 4. 
\begin{minipage}[t]{11.5 cm}
	In what order are class constructors and class destructors called when a derived-class object is created and deleted?
\end{minipage} \\[1ex]
\phantom{2. } 
\begin{minipage}[t]{11.5 cm}
	{\slshape 
	When the constructor is invoked for the derived class, the base 
	class constructor is is called first, and  then the derived 
	class constructor is called. 
	When the destructor is invoked for the derived class, the derived
	class destructor is called first and then the base class destructor
	is called. 
	} 
\end{minipage} 
\\[2ex]

\noindent 5. 
\begin{minipage}[t]{11.5 cm}
	If a derived class doesn't add any data members to the base class, does the derived class require constructors?
\end{minipage} \\[1ex]
\phantom{3. } 
\begin{minipage}[t]{11.5 cm}
	{\slshape 
	If the defined class does not necessarily require an explicit 
	constructor, because a default is provided in the event that one
	is not defined. 
	However, the default derived class constructor is a do nothing
	constructor that calls the default constructor for the base
	class. 
	If this is not what the user wants, it is best to specify
	a constructor for the derived class explicitly, because it
	gives the user control as to which constructor is called for
	the base class as well, even if the derived class constructor is
	do nothing.
	} 
\end{minipage} 
\\[2ex]

\noindent 6. 
\begin{minipage}[t]{11.5 cm}
	Suppose a base class and a derived class both define a method with the same name and a derived-class object invokes the method. What method is called?
\end{minipage} \\[1ex]
\phantom{3. } 
\begin{minipage}[t]{11.5 cm}
	{\slshape 
	There are two scenarios:
	\begin{enumerate}
		\item The methods are not virtual: \\
			If a pointer of type <base class> is pointing to the 
			derived class, invoking the method through the pointer will
			invoke the base class method. 
			If a reference of type <base class> is assigned to the
			derived class, the invoking method through the reference will
			invoke the base class method.
		\item The methods are vitual
			If a pointer of type <base class> is pointing to the 
			derived class, invoking the method through the pointer will
			invoke the derived class method. 
			If a reference of type <base class> is assigned to the
			derived class, the invoking method through the reference will
			invoke the derived class method.
	\end{enumerate}
	
	} 
\end{minipage} 
\\[2ex]

\noindent 7. 
\begin{minipage}[t]{11.5 cm}
	When should a derived class define an assignment operator?
\end{minipage} \\[1ex]
\phantom{2. } 
\begin{minipage}[t]{11.5 cm}
	{\slshape 
	An assignment operator should be explicitly defined in the case
	that the derived class object has a dynamically allocated object
	that was not inherited from the base class.
	The overloaded operator should do deep copying.
	It is also important to note that in this case the copy constructor
	should be explicitly defined as well.
	}  
\end{minipage} 
\\[2ex]

\noindent 8. 
\begin{minipage}[t]{11.5 cm}
	Can you assign the address of an object of a derived class to a pointer to the base class? Can you assign the address of an object of a base class to a pointer to the derived class?
\end{minipage} \\[1ex]
\phantom{3. } 
\begin{minipage}[t]{11.5 cm}
	{\slshape 
	Assigning the address of a derived class to a pointer to the
	base class is called upcasting and it may be done without an 
	explicit type cast.
	However, the opposite is called downcasting and can only be done
	with an explicit type cast. 
	} 
\end{minipage} 
\\[2ex]

\noindent 9. 
\begin{minipage}[t]{11.5 cm}
	Can you assign an object of a derived class to an object of the base class? Can you assign an object of a base class to an object of the derived class?
\end{minipage} \\[1ex]
\phantom{2. } 
\begin{minipage}[t]{11.5 cm}
	{\slshape 
	Because of implicit upcasting, you can assign an object of a
	derived class to an object of the base class. 
	Without an explicit type cast or conversion constructor, 
	you cannot assign an object of the derived class to an object
	of the base class.
	} 
\end{minipage} 
\\[2ex]

\noindent 10. 
\begin{minipage}[t]{11.5 cm}
	Suppose you define a function that takes a reference to a base-class object as an argument. Why can this function also use a derived-class object as an argument?
\end{minipage} \\[1ex]
\phantom{3. } 
\begin{minipage}[t]{11.5 cm}
	{\slshape 
	Yes, this is permissible because upcasting is implicit. 
	Since the derived class inherits member values and functions
	from the base class, any derived class methods will work
	when performed on the base class.
	} 
\end{minipage} 
\\[2ex]

\noindent 11. 
\begin{minipage}[t]{11.5 cm}
	Suppose you define a function that takes a base-class object as an argument (that is, the function passes a base-class object by value). Why can this function also use a derived-class object as an argument?
\end{minipage} \\[1ex]
\phantom{3. } 
\begin{minipage}[t]{11.5 cm}
	{\slshape 
	When a function's formal parameter is an object, the copy constructor
	is used to create a temporary object which is used in the body
	of the function. 
	Because of implicit upcasting, the copy constructor for the
	base class object may take a derived class object as its actual
	argument without any problems. 
	} 
\end{minipage} 
\\[2ex]

\noindent 12. 
\begin{minipage}[t]{11.5 cm}
	Why is it usually better to pass objects by reference than by value?
\end{minipage} \\[1ex]
\phantom{3. } 
\begin{minipage}[t]{11.5 cm}
	{\slshape 
	Passing by reference obviates the need to create a temporary
	object with the copy constructor and thus makes the program faster.
	} 
\end{minipage} 
\\[2ex]

\noindent 13. 
\begin{minipage}[t]{11.5 cm}
	Suppose \texttt{Corporation} is a base class and \texttt{PublicCorporation} is a derived class. Also suppose that each class defines a \texttt{head()} member function, that \texttt{ph} is a pointer to the \texttt{Corporation} type, and that \texttt{ph} is assigned the address of a \texttt{PublicCorporation} object. How is \verb+ph->head()+ interpreted if the base class defines \texttt{head()} as a 
\end{minipage} \\[1ex]
\phantom{1. }a.
\begin{minipage}[t]{11.5 cm}
	Regular nonvirtual method \\[1ex]
	{\slshape 
	The program runs \verb+Corporation::head()+.
	}  \\
	{} % space for comments to be included
\end{minipage} \\[1ex]
\phantom{1. }b.
\begin{minipage}[t]{11.5 cm}
	Virtual method \\[1ex]
	{\slshape 
	The program runs \verb+PublicCorporation::head()+.
	} \\
	{} % space for comments to be included
\end{minipage}
\\[2ex]

\noindent 14. 
\begin{minipage}[t]{11.5 cm}
	What's wrong, if anything, with the following code?
	\begin{verbatim}
		class Kitchen
		{
		private:
		    double kit_sq_ft;
		public:
		    Kitchen() {kit_sq_ft = 0.0; }
		    virtual double area() const { return kit_sq_ft * kit_sq_ft; }
		};
		class House : public Kitchen
		{
		private:
		    double all_sq_ft;
		public:
		    House() {all_sq_ft += kit_sq_ft;}
		    double area(const char *s) const { cout << s; return all_sq_ft; }
		};
	\end{verbatim}
\end{minipage} \\[1ex]
\phantom{3. } 
\begin{minipage}[t]{11.5 cm}
	{\slshape 
	Several things are wrong with this code.
	\begin{enumerate}
		\item 
		We have  a base class and a derived class, but the
		derived class relationship with the base class does not follow
		an \textbf{is-a} relationship.

		\item
		The variable \verb+all_sq_ft+ is never initialized, so
		we are in store for unpleasant surprises when we attempt to
		print or use the value.

		\item
		There is no point to making \verb+area()+ a virtual
		function if the \verb+area()+ function in the derived class
		has a different function signiature. 

		\item
		We need to include the \verb+iostream+ library and include
		a using declaration so that we may use \verb+cout+ in the
		\verb+area()+ function of the \verb+House+ object.

		\item	
		The base class should have a virtual destructor,
		which it does not.
	\end{enumerate}
	}  
\end{minipage} 
	
\end{document}

Here is the format for questions that include multiple parts:

% Two parts 
\noindent X. 
\begin{minipage}[t]{11.5 cm}
	The question
\end{minipage} \\[1ex]
\phantom{1. }a.
\begin{minipage}[t]{11.5 cm}
	part a \\[1ex]
	{\slshape 
	The answer.
	}  \\
	{} % space for comments to be included
\end{minipage} \\[1ex]
\phantom{1. }b.
\begin{minipage}[t]{11.5 cm}
	part b \\[1ex]
	{\slshape 
	The answer.
	} \\
	{} % space for comments to be included
\end{minipage}
\\[2ex]

% Three Parts
\noindent X. 
\begin{minipage}[t]{11.5 cm}
	The question
\end{minipage} \\[1ex]
\phantom{1. }a.
\begin{minipage}[t]{11.5 cm}
	part a \\[1ex]
	{\slshape The answer.} \\
	{} % space for comments to be included
\end{minipage} \\[1ex]
\phantom{1. }b.
\begin{minipage}[t]{11.5 cm}
	part b \\[1ex]
	{\slshape The answer}\\
	{} % space for comments to be included
\end{minipage} \\[1ex]
\phantom{1. }c.
\begin{minipage}[t]{11.5 cm}
	part c \\[1ex]
	{\slshape The answer} \\
	{} % space for comments to be included
\end{minipage}
\\[2ex]

% Four Parts
\noindent X. 
\begin{minipage}[t]{11.5 cm}
	The question
\end{minipage} \\[1ex]
\phantom{1. }a.
\begin{minipage}[t]{11.5 cm}
	part a \\[1ex]
	{\slshape The answer.} \\
	{} % space for comments to be included
\end{minipage} \\[1ex]
\phantom{1. }b.
\begin{minipage}[t]{11.5 cm}
	part b \\[1ex]
	{\slshape The answer}\\
	{} % space for comments to be included
\end{minipage} \\[1ex]
\phantom{1. }c.
\begin{minipage}[t]{11.5 cm}
	part c \\[1ex]
	{\slshape The answer} \\
	{} % space for comments to be included
\end{minipage} \\[1ex]
\phantom{1. }d.
\begin{minipage}[t]{11.5 cm}
	part d \\[1ex]
	{\slshape The answer} \\
	{} % space for comments to be included
\end{minipage}
\\[2ex]

% Five Parts
\noindent X. 
\begin{minipage}[t]{11.5 cm}
	The question
\end{minipage} \\[1ex]
\phantom{1. }a.
\begin{minipage}[t]{11.5 cm}
	part a \\[1ex]
	{\slshape The answer.} \\
	{} % space for comments to be included
\end{minipage} \\[1ex]
\phantom{1. }b.
\begin{minipage}[t]{11.5 cm}
	part b \\[1ex]
	{\slshape The answer}\\
	{} % space for comments to be included
\end{minipage} \\[1ex]
\phantom{1. }c.
\begin{minipage}[t]{11.5 cm}
	part c \\[1ex]
	{\slshape The answer} \\
	{} % space for comments to be included
\end{minipage} \\[1ex]
\phantom{1. }d.
\begin{minipage}[t]{11.5 cm}
	part d \\[1ex]
	{\slshape The answer} \\
	{} % space for comments to be included
\end{minipage} \\[1ex]
\phantom{1. }e.
\begin{minipage}[t]{11.5 cm}
	part e \\[1ex]
	{\slshape The answer} \\
	{} % space for comments to be included
\end{minipage}
\\[2ex]

% Six Parts
\noindent X. 
\begin{minipage}[t]{11.5 cm}
	The question
\end{minipage} \\[1ex]
\phantom{1. }a.
\begin{minipage}[t]{11.5 cm}
	part a \\[1ex]
	{\slshape The answer.} \\
	{} % space for comments to be included
\end{minipage} \\[1ex]
\phantom{1. }b.
\begin{minipage}[t]{11.5 cm}
	part b \\[1ex]
	{\slshape The answer}\\
	{} % space for comments to be included
\end{minipage} \\[1ex]
\phantom{1. }c.
\begin{minipage}[t]{11.5 cm}
	part c \\[1ex]
	{\slshape The answer} \\
	{} % space for comments to be included
\end{minipage} \\[1ex]
\phantom{1. }d.
\begin{minipage}[t]{11.5 cm}
	part d \\[1ex]
	{\slshape The answer} \\
	{} % space for comments to be included
\end{minipage} \\[1ex]
\phantom{1. }e.
\begin{minipage}[t]{11.5 cm}
	part e \\[1ex]
	{\slshape The answer} \\
	{} % space for comments to be included
\end{minipage} \\[1ex]
\phantom{1. }f.
\begin{minipage}[t]{11.5 cm}
	part f \\[1ex]
	{\slshape The answer} \\
	{} % space for comments to be included
\end{minipage}
\\[2ex]

\noindent X. 
\begin{minipage}[t]{11.5 cm}
	The question
\end{minipage} \\[1ex]
\phantom{3. } 
\begin{minipage}[t]{11.5 cm}
	{\slshape The answer.} 
\end{minipage} 
\\[2ex]
