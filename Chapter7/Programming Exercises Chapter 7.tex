\documentclass[10 pt]{amsart}

\usepackage{amssymb,latexsym}
\usepackage{graphicx}

% For the cpart environment, although it would probably be better in the
% future to implement this with a list environment.
\newlength{\cgap}
\settowidth{\cgap}{\qquad \textbf{x. }}
\newlength{\cwidth}
\setlength{\cwidth}{\textwidth}
\addtolength{\cwidth}{-\cgap}
\newenvironment{cpart}[2][\cwidth]
	{\\ \phantom{\qquad}\textbf{#2. }\begin{minipage}[t]{#1}}
	{\end{minipage}}

\newcommand{\ttt}[1]{\texttt{#1}}
\newcommand{\ttb}[1]{\pmb{\texttt{#1}}}
% Macros, all must be filled out
\newcommand{\ChapNum}{7}

\begin{document}

	\title
	[Chapter \ChapNum]
	{C++ Primer Plus, 5$^\text{th}$ Edition \\
	Programming Exercises \\
	Chapter \ChapNum}

	\maketitle

	\begin{cpart}{1}
		Write a program that repeatedly asks the user to enter pairs
		of numbers until at least one of the pair is \ttt{0}.
		For each pair, the program should use a function to calculate
		the harmonic mean of the numbers.
		The function should return the answer to \ttt{main()},
		which should report the result.
		The harmonic mean of the numbers is the inverse of the 
		average of the inverses and can be calculated as follows:\\[2ex]
		harmonic mean = $2.0 \times x \times y / (x + y)$
	\end{cpart}
	\vspace{2ex}

	\begin{cpart}{2}
		Write a program that asks the user to enter up to 10 golf
		scores, which are to be stored in an array.
		You should provide a means for the user to terminate input prior
		to entering 10 scores.
		The program should display all the scores on one line and report
		the average score.
		Handle input, display, and the average calculation with three
		separate array-processing functions.
	\end{cpart}
	\vspace{2ex}

	\begin{cpart}{3}
		Here is a structure declaration:
		{\ttfamily
			\begin{tabbing}
				\phantom{\qquad}\=\phantom{\qquad}\=\phantom{\qquad}\= \\
				struct box\\
				\{ \\
				\>	char maker[40]; \\
				\> float height; \\
				\> float width; \\
				\> float length;  \\
				\> float volume; \\
				\};
			\end{tabbing}
		}
		\vspace{2ex}
		\leftskip .25 cm
		\parindent -.5 cm
		a.
			Write a function that passes a \ttt{box} structure by 
			value and that displays the value of each member.\\

		b.
			Write a function that passes the address of a \ttt{box}
			structure and that sets the \ttt{volume} member to the product
			of the other three dimensions.\\

		c. 
			Write a simple program that uses these two functions.
	\end{cpart}
	\newpage

	\begin{cpart}{4}
		Many state lotteries use a variation of the simple lottery
		portrayed by Listing 7.4.
		In these variations you choose several numbers from one set
		and call them the field numbers.
		For example, you might select 5 numbers from the field of 
		1-47).
		You also pick a single number (called a mega number) or a
		power ball, etc.) from a second range, such as 1-27.
		To win the grand prize, you have to guess all the picks 
		correctly. 
		The chance of winning is the product of the probability of
		picking all the field numbers times the probability
		of picking the mega number.
		For instance, the probability of winning the example
		described here is the product of the probability of 
		picking 5 out of 47 correctly times the probability
		of picking 1 out of 27 correctly.
		Modify Listing 7.4 to calculate the probability
		of winning this kind of lottery.
	\end{cpart}
	\vspace{2ex}

	\begin{cpart}{5}
		Define a recursive function that takes an integer argument 
		and returns the factorial of that argument.
		Recall that 3 factorial, written 3!, equals $3 \times 2!$, 
		and so on, with 0! defined as 1. 
		In general, if $n$ is greater than 0, $n! = n*(n - 1)!$. 
		Test your function in a program that uses a loop to allow
		the user to enter various values for which the program
		reports the factorial.
	\end{cpart}
	\vspace{2ex}

	\begin{cpart}{6}
		Write a program that uses the following functions: \\[2ex]
		\ttt{Fill\_array()} takes as arguments the name of an array
		of \ttt{double} values and an array size.
		It prompts the user to enter \ttt{double} values to be
		entered into the array.
		It ceases taking input when the array is full or when
		the user enters non-numeric input, and it returns the 
		actual number of entries. \\[2ex]
		\ttt{Show\_array()} takes as arguments the name of an array of 
		\ttt{double} values and an array size and displays the
		contents of the array. \\[2ex]
		\ttt{Reverse\_array()} takes as arguments the name of an
		array of \ttt{double} values an an array size and reverses
		the order of the values stored in the array. \\[2ex]
		The program should use these functions to fill an array,
		show the array, reverse the array, show the array, 
		reverse all but the first and last elements of the array,
		and then show the array.
	\end{cpart}
	\vspace{2ex}

	\begin{cpart}{7}
		Redo Listing 7.7, modifying the three array-handling functions
		to each use two pointer parameters to represent a range.
		The \ttt{fill\_array()} function, instead of returning the 
		actual number of items read, should return a ponter to the
		location after the last location filled;
		the other functions can use this pointer as the second argument
		to identify the end of the data.
	\end{cpart}
	\vspace{2ex}

	\begin{cpart}{8}
		This exercise provides practice in writing functions dealing
		with arrays and structures.
		The following is a program skeleton.
		Complete it by providing the described functions:
		{\ttfamily
			\begin{tabbing}
				\phantom{\qquad}\=\phantom{\qquad}\=\phantom{\qquad}\= \\
				\#include <iostream> \\
				using namespace std; \\
				\\
				const int SLEN = 30; \\
				struct student \{ \\
				\>	char fullname[SLEN]; \\
				\> char hobby[SLEN]; \\
				\> int ooplevel; \\
				\}; \\
				// getinfo() has two arguments: a pointer to the first element
				of \\
				// an array of student structures and an int representing
				the \\
				// number of elements of the array. The function solicits
				and \\
				// stores data about students. It terminates input
				upon filling \\
				// the array or upon encountering a blank line for
				the student \\
				// name. The function returns the actual number of 
				array elements \\
				// filled
				int getinfo(student pa[], int n); \\
				\\
				// display1() takes a student structure as an argument \\
				// and displays its contents \\
				void display1(student st); \\
				\\
				// display2() takes the address of student structure as
				an \\
				// argument and displays the structure's contents \\
				void display2(const student * ps); \\
				\\
				// display3() takes the address of the first element
				of an array \\
				// of student structures and the number of array
				elements as \\
				// arguments and displays the contents of the structures \\
				void display3(const student pa[], int n); \\
				\\
				int main() \\
				\{ \\
				\>		cout << "Enter class size: "; \\
				\> 	int class\_size; \\
				\> 	cin >> class\_size; \\
				\> 	while (cin.get() != '\textbackslash n') \\
				\> \> 	continue; \\
				\\
				\> 	student * ptr\_stu = new student[class\_size]; \\
				\> 	int entered = getinfo(ptr\_stu, class\_size); \\
				\> 	for (int i = 0; i < entered; i++) \\
				\> 	\{ \\
				\> \> 	display1(ptr\_stu[i]); \\
				\> \> 	display2(\&ptr\_stu[i]); \\
				\> 	\} \\
				\> display3(ptr\_stu, entered); \\
				\> delete[] ptr\_stu; \\
				\> cout << "Done \textbackslash n"; \\
				\> return 0; \\
				\}
			\end{tabbing}
		}
	\end{cpart}
	\vspace{2ex}

	\begin{cpart}{9}
		Design a function \ttt{calculate()} that takes two type
		\ttt{double} values and a pointer to a function that takes
		two \ttt{double} arguments and returns a \ttt{double}.
		The \ttt{calculate()} function should also be type \ttt{double}, 
		and it should return teh value that the pointer-to function
		calculates, using the double arguements to \ttt{calculate()}.
		For example, suppose you have this definition for the 
		\ttt{add()} function:
		{\ttfamily
			\begin{tabbing}
				\phantom{\qquad}\=\phantom{\qquad}\=\phantom{\qquad}\= \\
				double add(double x, double y) \\
				\{ \\
				\> return x + y; \\
				\}
			\end{tabbing}
		}		
		Then the function call in \\[2ex]
		\ttt{double q = calculate(2.5, 10.4, add);} \\[2ex]
		would cause \ttt{calculate()} to pass the values
		\ttt{2.5} and \ttt{10.4} to the \ttt{add()} function
		and then return the \ttt{add()} return value \ttt{(12.9)}. \\[2ex]
		Use these functions and at least one additional function in the 
		\ttt{add()} mold in a program.
		The program should use a loop that allows the user to enter
		pairs of numbers.
		For each pair, use \ttt{calculate()} to invoke \ttt{add()}
		and at least one other function.
		If you are feeling adventurous, try creating an array of pointers
		to \ttt{add()}-style functions and use a loop to
		successively apply \ttt{calculate()} to a series of functions
		by using these pointers.
		Hint: Here's how to declare such an array of three pointers: \\[2ex]
		\ttt{double(*pf[3])(double, double);} \\[2ex]
		You can initialize such an array by using the usual array
		initialization syntax and function names and addresses.
	\end{cpart}
	\vspace{2ex}

\end{document}

regarding tabbing environments:
\= (set tab)
\> (advance to next tab stop)
\<
\+ (indent; move margin right)
\- (unindent; move margin left)
\'
\`
\\ (end of line; newline)
\kill (ignore preceding text; use only for spacing)



{\ttfamily
	\begin{tabbing}
		\phantom{\qquad}\=\phantom{\qquad}\=\phantom{\qquad}\= \\
		
	\end{tabbing}
}











