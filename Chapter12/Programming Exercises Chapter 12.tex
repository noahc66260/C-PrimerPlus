\documentclass[10 pt]{amsart}

\usepackage{amssymb,latexsym}
\usepackage{graphicx, setspace, enumerate}
% the setspace package allows us use of the 
% spacing environment (\begin{spacing}{second arg}) where
% second arg is a number to multiply to the spacing factor.
% use 2 for double space, 1 for single space, etc.

% For the cpart environment, although it would probably be better in the
% future to implement this with a list environment.
\newlength{\cgap}
\settowidth{\cgap}{\textbf{x. }}
\newlength{\cwidth}
\setlength{\cwidth}{\textwidth}
\addtolength{\cwidth}{-\cgap}
\newenvironment{cpart}[2][\cwidth]
	{%		
		\\ %
		\textbf{#2. }%
		\begin{minipage}[t]{#1}%
		\setlength{\parindent}{0pt}%
		\setlength{\parskip}{2ex}%
	}
	{%
		\end{minipage}%
	}
\newenvironment{cpartContinued}[2][\cwidth]
	{%		
		\\ %
		\textbf{#2. (continued)}%
		\\
		\phantom{#2. }
		\begin{minipage}[t]{#1}%
		\setlength{\parindent}{0pt}%
		\setlength{\parskip}{2ex}%
	}
	{%
		\end{minipage}%
	}


% set paragraph spacing like that in the book
\setlength{\parindent}{0pt}
\setlength{\parskip}{2ex}

% these new commands will make typing different formats easier.
\newcommand{\ttt}[1]{\texttt{#1}}
\newcommand{\ttb}[1]{\pmb{\texttt{#1}}}
\newcommand{\tbs}{\textbackslash}
% Macros, all must be filled out
\newcommand{\ChapNum}{12}

\begin{document}
	\title
	[Chapter \ChapNum]
	{C++ Primer Plus, 5$^\text{th}$ Edition \\
	Programming Exercises \\
	Chapter \ChapNum}

	\maketitle

	\begin{cpart}{1}
		Consider the following class declaration:
		{\ttfamily
			\begin{tabbing}
				\phantom{\qquad}\= \\
				class Cow \{ \\
				\> 	char name[20]; \\
				\> 	char * hobby; \\
				\> 	double weight; \\
				public: \\
				\> 	Cow(); \\	
				\> 	Cow(const char * nm, const char * ho, double wt); \\
				\> 	Cow(const Cow \& c); \\
				\> 	$\sim$Cow(); \\
				\> 	Cow \& operator==(const Cow \& c); \\
				\> 	void ShowCow() const; // display all cow data \\
				\}; 
			\end{tabbing}
		}
		Provide teh implementations for this class and write a short
		program that uses all member functions.
	\end{cpart}
	\newpage

	\begin{cpart}{2}
		Enhance the \ttt{String} class declaration (that is, upgrade
		\ttt{string1.h} to \ttt{string2.h}) by doing the following:
		\begin{enumerate}[a.]
			\item 
				Overload the \ttt{+} operator to allow you to 
				join two strings into one.
			\item 
				Provide a \ttt{stringlow()} member function that converts
				all alphabetic characters in a string to lowercase.
				(Don't forget the \ttt{cctype} family of character
				functions.)
			\item
				Provide a \ttt{stringup()} member function that converts
				all alphabetic characters in a string to uppercase.
			\item
				Provide a member function that takes a \ttt{char}
				argument and returns the number of times the character
				appears in a string.
		\end{enumerate}
		Test your work in the following program:
		{\ttfamily 
			\begin{tabbing}
				\phantom{\qquad}\=\phantom{\qquad}\=\phantom{\qquad}\=
				\hspace{1cm}\=\hspace{3cm}\= \\
				// pe12\_2.cpp \\
				\#include <iostream> \\
				using namespace std; \\
				\#include "string2.h" \\
				int main() \\
				\{
				\+ \\
				String s1(" and I am a C++ student."); \\
				String s2 = "Please enter your name: "; \\
				String s3; \\
				cout << s2; \> \> \> \> // overloaded << operator \\
				cin >> s3;  \> \> \> \> // overloaded >> operator \\
				s2 = "My name is " + s3; \> \> \> \>
					// overloaded =, + operators \\
				cout << s2 << "\tbs n"; \\
				s2 = s2 + s1; \\
				s2.stringup(); \> \> \> \> \> 
					// converts string to uppercase \\
				cout << "The string\tbs n" << s2 << "\tbs ncontains "
						<< s2.has('A') \\
				\phantom{cout} << " 'A' characters in it.\tbs n"; \\
				s1 = "red"; \> \> \> // String(const char *), \\
				\> \> \> // then String & operator=(const String &) \\
				String rgb[3] = \{ String(s1), 
					String("green"), String("blue")\}; \\
				cout << "Enter the name of a primary color for 
							mixing light: "; \\
				String ans; \\
				bool success = false; \\
			\end{tabbing}
		}
		\end{cpart}
		\newpage

		\begin{cpartContinued}{2}
		{\ttfamily 
			\begin{tabbing}
				\phantom{\qquad}\=\phantom{\qquad}\=\phantom{\qquad}\=
				\hspace{1cm}\=\hspace{3cm}\= 
				\+ \\
				while (cin >> ans) \\
				\{ 
				\+ \\
				ans.stringlow(); \> \> \> // converts string to lowercase \\
				for (int i = 0; i < 3; i++) \\
				\{
				\+ \\
				if (ans == rgb[i]) \> \> // overloaded == operator \\
				\{
				\+ \\
				cout << "That's right!\tbs n"; \\
				success = true; \\
				break; \\
				\} 
				\- \\
				\} \\
				if (success) \\
				\> 	break; \\
				else \\
				\> 	cout << "Try again!\tbs n";
				\- \\
				\} \\
				cout << "Bye\tbs n"; \\
				return 0;
				\- \\
				\}
			\end{tabbing}
		}
		Your output should look like this sample run:

		{\ttfamily
			Please enter your name:\ttb{ Fretta Farbo} \\
			My name is Fretta Farbo. \\
			The string \\
			MY NAME IS FRETTA FARBO AND I AM A C++ STUDENT. \\
			contains 6 'A' characters in it. \\
			Enter the name of a primary color for mixing light;\ttb{ yellow} \\
			Try again! \\
			\ttb{BLUE} \\
			That's right! \\
			Bye
		}
	\end{cpartContinued}

	\begin{cpart}{3}
		Rewrite the \ttt{Stock} class, as described in Listings 10.7
		and 10.8 in chapter 10, so that it uses dynamically allocated
		memory directly instead of using \ttt{string} class objects
		to hold the stock names.
		Also, replace the \ttt{show()} member function with an overloaded
		\ttt{operator<<()} definition.
		Test the new definition program in Listing 10.9.
	\end{cpart}

	\begin{cpart}{4}
		Consider the following variation of the \ttt{Stack} class 
		defined in Listing 10.10:
		{\ttfamily
			\begin{tabbing}
				\phantom{\qquad}\=\hspace{4cm}\= \\
				// stack.h -- class declaration for the stack ADT \\
				typedef unsigned long Item; \\
				\\
				class Stack \\
				\{ \\
				private: 
				\+ \\
				enum {MAX = 10}; \> // constant specific to class \\
				Item * pitems; \> // holds stack items \\
				int size; \> // number of elements in stack \\
				int top; \> // index for top item of stack \\
				\< public: \\
				Stack(int n = 10); \> // creates stack with n elements \\
				Stack(const Stack & st); \\
				$\sim$Stack(); \\
				bool isempty() const; \\
				bool isfull() const; \\
				// push() returns false if stack already is full,
					true otherwise \\
				bool push(const Item \& item); // add item to stack \\
				// pop() returns false if stack already is empty, 
					true otherwise \\
				bool pop(Item \& item); // pop top into item \\
				Stack & operator=(const Stack \& st); \\
				\< \};
			\end{tabbing}
		}		
		As the private members suggest, this class uses a dynamically
		allocated array to thold the stack items.
		Rewrite the methods to fit this new representation and 	
		write a program that demonstrates all the methods, including
		the copy constructor and assignment operator.
	\end{cpart}

	\begin{cpart}{5}
		The Bank of Heather has performed a study showing that ATM 
		customers won't wait more than one minute in line.
		Using the simulation from Listing 12.10, find a value for
		the number of customers per hour that leads to an
		average wait time of one minute. 
		(Use at least a 100-hours trial period.)
	\end{cpart}

	\begin{cpart}{6}
		The Bank of Heather would like to know what would happen if it 
		added a second ATM.
		Modify the simulation in this chapter so that it has two
		queues.
		Assume that a customer will join the first queue if it has
		fewer people in it than the second queue and that
		the customer will joint the second queue otherwise.
		Again, find a value for the number of customers per hour
		that leads to an average wait time of one minute.
		(Note: This is a non-linear problem in that doubling the
		number of ATMs doesn't double the number of customers who can
		be handled per hour with a one-minute wiat maximum.)
	\end{cpart}

\end{document}

regarding tabbing environments:
\= (set tab)
\> (advance to next tab stop)
\<
\+ (indent; move margin right)
\- (unindent; move margin left)
\'
\`
\\ (end of line; newline)
\kill (ignore preceding text; use only for spacing)

use \hspace{...} if you prefer

		{\ttfamily
			\begin{tabbing}
				\phantom{\qquad}\=\phantom{\qquad}\=\phantom{\qquad}\= \\
		
			\end{tabbing}
		}











