\documentclass{amsart}

\usepackage{amssymb,latexsym, verbatim}
\thispagestyle{empty}
\pagestyle{empty}

\begin{document}
\begin{center}
	\Large {\bfseries
	\emph{C++ Primer Plus, $5^{\text{th}}$ Edition} by Stephen Prata \\
	Chapter 15: Friends, Exceptions, and More \\
	Review Questions} \normalsize \vspace{5ex}
\end{center}

% Note: in order to really get the formatting I want, I need to create my own environment. It would be similar to the enumerate environment, but instead of enclosing the enumerator in parentheses, there would just be a period after it. For example, rather than (1), we would have 1. instead.

\phantom{\quad}\vfill
\noindent 1. 
\begin{minipage}[t]{11.5 cm}
	What's wrong with the following attempts at establishing friends?
\end{minipage} \\[1ex]
\phantom{1. }a.
\begin{minipage}[t]{11.5 cm}
	\begin{verbatim}
		class snap {
		    friend clasp;
		    ...
		};
		class clasp { ... };
	\end{verbatim}	
	{\slshape 
		The line of code \verb+friend clasp+ in the \verb+snap+ class
		should be \verb+friend class clasp+ so the compiler knows that
		\verb+clasp+ is a class. 
	} \\
	{} % space for comments to be included
\end{minipage} \\
\phantom{1. }b.
\begin{minipage}[t]{11.5 cm}
	\begin{verbatim}
		class cuff {
		public:
		    void snip(muff &) { ... }
		    ...
		};
		class muff {
		    friend void cuff::snip(muff &);
		    ...
		};
	\end{verbatim}
	{\slshape 
		The \verb+cuff+ class uses a \verb+muff+ reference in as a
		formal argument in a member function.
		However, the compiler has no way of knowing what type \verb+muff+
		is.
		To solve this problem, we should insert a forward declaration
		before we define the \verb+cuff+ class that reads
		\verb+class muff;+
	}\\
	{} % space for comments to be included
\end{minipage} \\
\phantom{1. }c. 
\begin{minipage}[t]{11.5 cm}
	\begin{verbatim}
		class muff {
		    friend void cuff::snip(muff &);
		    ...
		};
		class cuff {
		public:
		    void snip(muff &) { ... }
		    ...
		};
	\end{verbatim}
	{\slshape 
		The complier needs to see the definition of the \verb+cuff+ class
		before we can declare its functions friends to another class. 
		We can't solve this by a forward declaration at the top since
		the definition of the \verb+cuff+ class must precede that of 
		the \verb+muff+ class. 
		However, this this suffers the same problem as part (b), which 
		can be addresses by placing a forward declaration for the
		\verb+muff+ class at the top.
	} \\
	{} % space for comments to be included
\end{minipage}
\vfill
\newpage

\phantom{\quad}\vfill
\noindent 2. 
\begin{minipage}[t]{11.5 cm}
	You've seen how to create mutual class friends. Can you create a more restricted form of friendship in which only some members of Class B are friends to Class A and some members of A are friends to B? Explain.
\end{minipage} \\[1ex]
\phantom{2. } 
\begin{minipage}[t]{11.5 cm}
	{\slshape 
		This is a tricky situation. 
		The compiler requires that we place a class definition
		before specific functions from that class are declared 
		as friends to another class. 
		However, if we want to make both classes such that each 
		class has functions that are friends of the other class,
		we clearly have a problem.
		A forward declaration is only helpful when we use a type
		that is of a class not declared yet;
		it is of no help if we wish to declare a class's functions
		as friends of another class before the original class
		is defined. 
		So as we can see, we can only have one class have member
		functions which are friends to the other class. 
		Otherwise, it is best to make two classes mutual friends. 
	} 
\end{minipage} 
\vfill

\noindent 3. 
\begin{minipage}[t]{11.5 cm}
	What problems might the following nested class declaration have?
	\begin{verbatim}
		class Ribs
		{
		private:
		    class Sauce
		    {
		        int soy;
		        int sugar;
		    public:
		        Sauce(int s1, int s2) : soy(s1), sugar(s2) { }
		    };
		    ...
		};
	\end{verbatim}
\end{minipage} \\[1ex]
\phantom{3. } 
\begin{minipage}[t]{11.5 cm}
	{\slshape 
		The problem is that we have no way of accessing the 
		variables \verb+soy+ and \verb+sugar+ of the 
		\verb+Sauce+ class once we construct an object
		of that class.
		The private variables of the \verb+Sauce+ class can only 
		be accessed by the \verb+Sauce+ class directly and
		can only be accessed by the \verb+Ribs+ class indirectly
		through the public member functions of the \verb+Sauce+ class.
		In this case, there are none which return the values 
		\verb+soy+ and \verb+sugar+.
	} 
\end{minipage} 
\vfill

\noindent 4. 
\begin{minipage}[t]{11.5 cm}
	How does \texttt{throw} differ from \texttt{return}?
\end{minipage} \\[1ex]
\phantom{2. } 
\begin{minipage}[t]{11.5 cm}
	{\slshape 
		When throw is called, control is moved to successive calling 
		functions until the first \verb+try-catch+ block is reached
		which matches the exception type thrown, or else the program
		aborts because of an unexpected exception.  
		When return is called control goes directly to the calling
		function.
		Both share the property that automatic variables declared
		after the function that the control is given to are deallocated 
		from memory.
	} 
\end{minipage} 
\vfill
\newpage

\phantom{\quad}\vfill
\noindent 5. 
\begin{minipage}[t]{11.5 cm}
	Suppose you have a hierarchy of exception classes that are derived from a base exception class. In what order should you place \texttt{catch} blocks?
\end{minipage} \\[1ex]
\phantom{3. } 
\begin{minipage}[t]{11.5 cm}
	{\slshape 
		\verb+catch+ blocks should be placed in decending order of 
		derivation. 
		That is, \verb+catch+ blocks whose argument is a derived class
		should be placed before the \verb+catch+ block of a base class.
	} 
\end{minipage} 
\vfill

\noindent 6. 
\begin{minipage}[t]{11.5 cm}
	Consider the \texttt{Grand}, \texttt{Superb}, and \texttt{Magnificent} classes defined in this chapter. Suppose \texttt{pg} is a type \texttt{Grand} * pointer that is assigned the address of an object of one of these three classes and that \texttt{ps} is a type \texttt{Superb} * pointer. What is the difference in how the following two code samples behave?
	\begin{verbatim}
		if (ps = dynamic_cast<Superb *>(pg))
		    ps->say();   // sample #1

		if (typeid(*pg) == typeid(Superb))
		    (Superb *) pg->say();   // sample #2
	\end{verbatim}
\end{minipage} \\[1ex]
\phantom{3. } 
\begin{minipage}[t]{11.5 cm}
	{\slshape 
		The first code sample will execute the statement
		\verb+ps->say();+ if \verb+pg+ points to a Superb or
		Magnificent object. 
		Since the \verb+say()+ function is virtual, the function
		call will correspond to the object \verb+pg+ points to.
		In the event that \verb+pg+ points to a \verb+Grand+ object,
		the test statement will evaluate as false and nothing
		will happen.

		The second code sample will execute the \verb+say()+ member
		function of the object \verb+pg+ points to only if 
		\verb+pg+ points to a \verb+Superb+ object. 
	}
\end{minipage} 
\vfill

\noindent 7. 
\begin{minipage}[t]{11.5 cm}
	How is the \verb+static_cast+ operator different from the \verb+dynamic_cast+ operator?
\end{minipage} \\[1ex]
\phantom{2. } 
\begin{minipage}[t]{11.5 cm}
	{\slshape 
		The operators have the form \\
		\verb+dummy_cast <+ \texttt{type-name}
		\verb+> (+\texttt{expression}\verb+)+. \\
		The \verb+static_cast+ operator will make the conversion if 
		the expression can be converted to the type specified by
		type-name implicitly or vica versa. 
		The \verb+dynamic_cast+ operator will allow us to assign a
		pointer to an object only if the expression can be implicitly
		converted to the type-specified by type-name.
	} 
\end{minipage} 
\vfill

\end{document}

Here is the format for questions that include multiple parts:

% Two parts 
\noindent X. 
\begin{minipage}[t]{11.5 cm}
	The question
\end{minipage} \\[1ex]
\phantom{1. }a.
\begin{minipage}[t]{11.5 cm}
	part a \\[1ex]
	{\slshape The answer.} \\
	{} % space for comments to be included
\end{minipage} \\[1ex]
\phantom{1. }b.
\begin{minipage}[t]{11.5 cm}
	part b \\[1ex]
	{\slshape The answer}\\
	{} % space for comments to be included
\end{minipage}
\\[2ex]

% Three Parts
\noindent X. 
\begin{minipage}[t]{11.5 cm}
	The question
\end{minipage} \\[1ex]
\phantom{1. }a.
\begin{minipage}[t]{11.5 cm}
	part a \\[1ex]
	{\slshape The answer.} \\
	{} % space for comments to be included
\end{minipage} \\[1ex]
\phantom{1. }b.
\begin{minipage}[t]{11.5 cm}
	part b \\[1ex]
	{\slshape The answer}\\
	{} % space for comments to be included
\end{minipage} \\[1ex]
\phantom{1. }c.
\begin{minipage}[t]{11.5 cm}
	part c \\[1ex]
	{\slshape The answer} \\
	{} % space for comments to be included
\end{minipage}
\\[2ex]

% Four Parts
\noindent X. 
\begin{minipage}[t]{11.5 cm}
	The question
\end{minipage} \\[1ex]
\phantom{1. }a.
\begin{minipage}[t]{11.5 cm}
	part a \\[1ex]
	{\slshape The answer.} \\
	{} % space for comments to be included
\end{minipage} \\[1ex]
\phantom{1. }b.
\begin{minipage}[t]{11.5 cm}
	part b \\[1ex]
	{\slshape The answer}\\
	{} % space for comments to be included
\end{minipage} \\[1ex]
\phantom{1. }c.
\begin{minipage}[t]{11.5 cm}
	part c \\[1ex]
	{\slshape The answer} \\
	{} % space for comments to be included
\end{minipage} \\[1ex]
\phantom{1. }d.
\begin{minipage}[t]{11.5 cm}
	part d \\[1ex]
	{\slshape The answer} \\
	{} % space for comments to be included
\end{minipage}
\\[2ex]

% Five Parts
\noindent X. 
\begin{minipage}[t]{11.5 cm}
	The question
\end{minipage} \\[1ex]
\phantom{1. }a.
\begin{minipage}[t]{11.5 cm}
	part a \\[1ex]
	{\slshape The answer.} \\
	{} % space for comments to be included
\end{minipage} \\[1ex]
\phantom{1. }b.
\begin{minipage}[t]{11.5 cm}
	part b \\[1ex]
	{\slshape The answer}\\
	{} % space for comments to be included
\end{minipage} \\[1ex]
\phantom{1. }c.
\begin{minipage}[t]{11.5 cm}
	part c \\[1ex]
	{\slshape The answer} \\
	{} % space for comments to be included
\end{minipage} \\[1ex]
\phantom{1. }d.
\begin{minipage}[t]{11.5 cm}
	part d \\[1ex]
	{\slshape The answer} \\
	{} % space for comments to be included
\end{minipage} \\[1ex]
\phantom{1. }e.
\begin{minipage}[t]{11.5 cm}
	part e \\[1ex]
	{\slshape The answer} \\
	{} % space for comments to be included
\end{minipage}
\\[2ex]

% Six Parts
\noindent X. 
\begin{minipage}[t]{11.5 cm}
	The question
\end{minipage} \\[1ex]
\phantom{1. }a.
\begin{minipage}[t]{11.5 cm}
	part a \\[1ex]
	{\slshape The answer.} \\
	{} % space for comments to be included
\end{minipage} \\[1ex]
\phantom{1. }b.
\begin{minipage}[t]{11.5 cm}
	part b \\[1ex]
	{\slshape The answer}\\
	{} % space for comments to be included
\end{minipage} \\[1ex]
\phantom{1. }c.
\begin{minipage}[t]{11.5 cm}
	part c \\[1ex]
	{\slshape The answer} \\
	{} % space for comments to be included
\end{minipage} \\[1ex]
\phantom{1. }d.
\begin{minipage}[t]{11.5 cm}
	part d \\[1ex]
	{\slshape The answer} \\
	{} % space for comments to be included
\end{minipage} \\[1ex]
\phantom{1. }e.
\begin{minipage}[t]{11.5 cm}
	part e \\[1ex]
	{\slshape The answer} \\
	{} % space for comments to be included
\end{minipage} \\[1ex]
\phantom{1. }f.
\begin{minipage}[t]{11.5 cm}
	part f \\[1ex]
	{\slshape The answer} \\
	{} % space for comments to be included
\end{minipage}
\\[2ex]

\noindent X. 
\begin{minipage}[t]{11.5 cm}
	The question
\end{minipage} \\[1ex]
\phantom{3. } 
\begin{minipage}[t]{11.5 cm}
	{\slshape The answer.} 
\end{minipage} 
\\[2ex]
