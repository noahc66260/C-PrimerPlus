\documentclass[10 pt]{amsart}

\usepackage{amssymb,latexsym}
\usepackage{graphicx, setspace, enumerate}
% the setspace package allows us use of the 
% spacing environment (\begin{spacing}{second arg}) where
% second arg is a number to multiply to the spacing factor.
% use 2 for double space, 1 for single space, etc.

% For the cpart environment, although it would probably be better in the
% future to implement this with a list environment.
\newlength{\cgap}
\settowidth{\cgap}{\textbf{x. }}
\newlength{\cwidth}
\setlength{\cwidth}{\textwidth}
\addtolength{\cwidth}{-\cgap}
\newenvironment{cpart}[2][\cwidth]
	{%		
		\\ %
		\textbf{#2. }%
		\begin{minipage}[t]{#1}%
		\setlength{\parindent}{0pt}%
		\setlength{\parskip}{2ex}%
	}
	{%
		\end{minipage}%
	}
\newenvironment{cpartContinued}[2][\cwidth]
	{%		
		\\ %
		\textbf{#2. (continued)}%
		\\
		\phantom{#2. }
		\begin{minipage}[t]{#1}%
		\setlength{\parindent}{0pt}%
		\setlength{\parskip}{2ex}%
	}
	{%
		\end{minipage}%
	}


% set paragraph spacing like that in the book
\setlength{\parindent}{0pt}
\setlength{\parskip}{2ex}

% these new commands will make typing different formats easier.
\newcommand{\ttt}[1]{\texttt{#1}}
\newcommand{\ttb}[1]{\pmb{\texttt{#1}}}
\newcommand{\tbs}{\textbackslash}
% Macros, all must be filled out
\newcommand{\ChapNum}{16}

\begin{document}
	\title
	[Chapter \ChapNum]
	{C++ Primer Plus, 5$^\text{th}$ Edition \\
	Programming Exercises \\
	Chapter \ChapNum}

	\maketitle

	\begin{cpart}{1}
		A {\bfseries \textsl{palindrome}} is a string that is the
		same backward as it is forward.
		Write a program that lets a user enter a string and that 
		passes to a \ttt{bool} function a reference to the string.
		The function should return \ttt{true} if the string is a
		palindrome and \ttt{false} otherwise.
		At this point, don't worry about complications such as
		capitalization, spaces, and punctuation.
		That is, this simple version should reject "Otto" and 
		"Madam, I'm Adam."
		Feel free to scan the list of string methods in Appendix F
		for methods to simplify the task.
	\end{cpart}

	\begin{cpart}{2}
		Do the same problem as given in Programming Exercise 1, but 
		do worry about complications such as capitalization, spaces,
		and punctuation. 
		That is, "Madam, I'm Adam" should test as a palindrome.
		For example, the testing function could reduce the string to 
		"madamimadam" and then test wheter the reverse is the same.
		Don't forget the useful \ttt{cctype} library.
		You might find an STL function or two useful although not 
		necessary.
	\end{cpart}

	\begin{cpart}{3}
		Redo Listing 16.3 so that it gets its words from a file.
		One approach is to use a \ttt{vector<string>} object instead
		of an array of \ttt{string}.
		Then, you can use the \ttt{push\_back()} to copy
		how ever many words are in your data file into the
		\ttt{vector<string>} object and use the \ttt{size()}
		member to determine the length of the word list.
		Because the program should read one word at a time from the
		file, you should use the \ttt{>>} operator rather than
		\ttt{getline()}.
		The file itself should contain words separated by spaces,
		tabs, or new lines.
	\end{cpart}

	\begin{cpart}{4}
		Write a function with an old-style interface that has
		the prototype:

		\ttt{int reduce(long ar[], int n);} 

		The actual arguments should be the name of an array and the 
		number of elements in the array.
		The function should sort an array, remove duplicate values,
		and return a value equal to the number of elements in the
		reduced array.
		Write the function using STL functions.
		(If you decide to use the general \ttt{unique()} function, 
		note that it returns the end of the resulting range.)
		Test the function in a short program.
	\end{cpart}

	\begin{cpart}{5}
		Do the same problem as described in Programming Exercise 4,
		except make it a template function: 
		
		{\ttfamily
			template <class T> \\
			int reduce(T ar[], int n);
		}

		Test the function in a short program, using both a \ttt{long}
		instantiation and a \ttt{string} instantiation.
	\end{cpart}

	\begin{cpart}{6}
		Redo the example shown in Listing 12.15, using the STL \ttt{queue}
		template class instead of the \ttt{Queue} class described
		in Chapter 12.

		\emph{Note: There is no listing 12.15. I believe the
		question is actually referring to Listing 12.12}
	\end{cpart}

	\begin{cpart}{7}
		A common game is the lottery card.
		The card has numbered spots of which a certain number are
		selected at random.
		Write a \ttt{Lotto()} function that takes two arguments.
		The first should be the number of spots on the lottery card,
		and the second should be the number of spots selected at 
		random.
		The function should return a \ttt{vector<int>} object
		that contains, in sorted order, the nubmers selected
		at random.
		For example, you could use the function as follows:

		{\ttfamily
			vector<int> winners; \\
			winners = Lotto(51,6);
		}

		This would assign to \ttt{winners} a vector that contains
		six numbers selected randomly from
		the range 1 through 51.
		Note that simply using \ttt{rand()} doesn't quite do the job
		because it may produce duplicate values.
		Suggestion: Have the function create a vector that contains
		all the possible values, use \ttt{random\_shuffle}, and then
		use the beginning of the shuffled vector to obtain the values.
		Also write a short program that lets you test the function.
	\end{cpart}

	\begin{cpart}{8}
		Mat and Pat want to invite their friends to a party.
		They ask you to write a program that does the following:
		\begin{itemize}
			\item 
				Allows Mat to enter a list of his friends' names.
				The names are stored in a container and then displayed in
				sorted order.
			\item
				Allows Pat to enter a list of her friend's names.
				The names are stored in a secon dcontainer and then
				displayed in sorted order.
			\item
				Creates a third container that merges the two 
				lists, eliminates dupicates, and displays the
				contents of this container.
		\end{itemize}
	\end{cpart}

\end{document}

regarding tabbing environments:
\= (set tab)
\> (advance to next tab stop)
\<
\+ (indent; move margin right)
\- (unindent; move margin left)
\'
\`
\\ (end of line; newline)
\kill (ignore preceding text; use only for spacing)

use \hspace{...} if you prefer

		{\ttfamily
			\begin{tabbing}
				\phantom{\qquad}\=\phantom{\qquad}\=\phantom{\qquad}\= \\
		
			\end{tabbing}
		}











