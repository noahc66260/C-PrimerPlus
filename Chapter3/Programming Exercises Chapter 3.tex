\documentclass[10 pt]{amsart}

\usepackage{amssymb,latexsym}
\usepackage{graphicx}

% For the cpart environment, although it would probably be better in the
% future to implement this with a list environment.
\newlength{\cgap}
\settowidth{\cgap}{\qquad \textbf{x. }}
\newlength{\cwidth}
\setlength{\cwidth}{\textwidth}
\addtolength{\cwidth}{-\cgap}
\newenvironment{cpart}[2][\cwidth]
	{\\ \phantom{\qquad}\textbf{#2. }\begin{minipage}[t]{#1}}
	{\end{minipage}}

% Macros, all must be filled out
\newcommand{\ChapNum}{3}

\begin{document}

	\title
	[Chapter \ChapNum]
	{C++ Primer Plus, 5$^\text{th}$ Edition \\
	Programming Exercises \\
	Chapter \ChapNum}

	\maketitle

	\begin{cpart}{1}
		Write a short program that asks for your height in integer inches 
		and then converts your height to feet and inches. 
		Have the Program use the underscore character to indicate where 
		to type the response. Also, use a \texttt{const} symbolic 
		constant to represent the conversion factor.
	\end{cpart}
	\vspace{2ex}

	\begin{cpart}{2}
		Write a short program that asks for your height in feet and inches
		and your weight in pounds.
		(Use three variables to store the information.)
		Have the program report your body mass index (BMI).
		To calculate the BMI, first convert your height in feet and
		inches to your height in inches (1 foot = 12 inches). 
		Then, convert your height in inches to your height in meters by
		multiplying by 0.0254. 
		Then, convert your weight in pounds into your mass in 
		kilograms by diving by 2.2. 
		Finally, compute your BMI by dividing your mass in kilograms by
		the square of your height in meters. 
		Use symbolic constants to represent the various conversion factors.
	\end{cpart}
	\vspace{2ex}

	\begin{cpart}{3}
		Write a program that asks the user to enter a latitude in degrees,
		minutes, and seconds and that then displays the latitude in
		decimal format.
		There are 60 seconds of arc to a minute and 60 minutes of arc
		to a degree;
		represent these values with symbolic constants.
		You should use a separate variable for each input value.
		A sample run should look like this:
		{\ttfamily
			\begin{tabbing}
				\phantom{\qquad}\=\phantom{\qquad}\=\phantom{\qquad}\= \\
				Enter a latitude in degrees, minutes, and 
					seconds: \\
				First, enter the degrees: {\bf 37} \\
				Next, enter the minutes of arc: {\bf 51} \\
				Finally, enter the seconds of arc: {\bf 19} \\
				37 degrees, 51 minutes, 19 seconds = 37.8553 degrees
			\end{tabbing}
		}
	\end{cpart}
	\vspace{2ex}

	\begin{cpart}{4}
		Write a program that asks the user to enter the number of seconds
		as an integer value (use type \textbf{long}) and that then
		displays the equivalent time in days, hours, minutes, and seconds. 
		Use symbolic constants to represent the number of hours in the day,
		the number of minutes in an hour, and the number of seconds in
		a minute. 
		The output should look like this:
		{\ttfamily
			\begin{tabbing}
				\phantom{\qquad}\=\phantom{\qquad}\=\phantom{\qquad}\= \\
				Enter the number of seconds: {\bf 31600000} \\
				31600000 seconds = 365 days, 46 minutes, 40 seconds
			\end{tabbing}
		}
	\end{cpart}
	\vspace{2ex}

	\begin{cpart}{5}
		Write a program that asks how many miles you have driven and how
		many gallons of gasoline you have used and then reports the
		miles per gallon your car has gotten. 
		Or, if you prefer, the program can request distance in kilometers
		and petrol in liters and then report the result European style,
		in liters per 100 km.
	\end{cpart}
	\vspace{2ex}

	\begin{cpart}{6}
		Write a program that asks you to enter an automobile gasoline
		consumption figure in the European style (liters per 100
		kilometers) and converts to the U.S. style of miles per 
		gallon.
		Note that in addition to using different units of measurement,
		the U.S. approach (distance / fuel) is the inverse of the 
		European approach (fuel / distance).
		Note that 100 kilometers is 62.14 miles, and 1 gallon is 
		3.875 liters.
		Thus, 19 mpg is about 12.4 1/100 km, and 27 mpg is 
		8.7 1/100 km.
	\end{cpart}
	\vspace{2ex}

\end{document}

regarding tabbing environments:
\= (set tab)
\> (advance to next tab stop)
\<
\+ (indent; move margin right)
\- (unindent; move margin left)
\'
\`
\\ (end of line; newline)
\kill (ignore preceding text; use only for spacing)



{\ttfamily
	\begin{tabbing}
		\phantom{\qquad}\=\phantom{\qquad}\=\phantom{\qquad}\= \\
		
	\end{tabbing}
}











