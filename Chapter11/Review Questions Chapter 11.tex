\documentclass{amsart}

\usepackage{amssymb,latexsym, verbatim}
\thispagestyle{empty}
\pagestyle{empty}

\begin{document}
\begin{center}
	\Large {\bfseries
	\emph{C++ Primer Plus, $5^{\text{th}}$ Edition} by Stephen Prata \\
	Chapter 11: Working with Classes \\
	Review Questions} \normalsize \vspace{5ex}
\end{center}

% Note: in order to really get the formatting I want, I need to create my own environment. It would be similar to the enumerate environment, but instead of enclosing the enumerator in parentheses, there would just be a period after it. For example, rather than (1), we would have 1. instead.

\phantom{\quad}
\vfill
\noindent 1. 
\begin{minipage}[t]{11.5 cm}
	Using a member function to overload the multiplication operator for the \texttt{Stonewt} class; have the operator multiply the data members by a type \texttt{double} value. Note that this will require carryover for the stone-pound representation. That is, twice 10 stone 8 pounds is 21 stone 2 pounds. 
\end{minipage} \\[1ex]
\phantom{3. } 
\begin{minipage}[t]{11.5 cm}
	{\slshape 
	See the following code:
	} 
	\begin{verbatim}
	// prototype
	Stonewt operator*(double x) const;

	// definition
	Stonewt Stonewt::operator*(double x) const
	{
	    double total_lbs = Lbs_per_stn*stn + lbs;
	    return Stonewt(x * total_lbs);
	}
	\end{verbatim}
\end{minipage} 
\vfill

\noindent 2. 
\begin{minipage}[t]{11.5 cm}
	What are the differences between a friend function and a member function?
\end{minipage} \\[1ex]
\phantom{2. } 
\begin{minipage}[t]{11.5 cm}
	{\slshape 
	When overloading operators, specifically a binary operator, using 
	a member function requires that the first operand be
	the object which invokes the function.
	If you use a friend, however, you may overload the 
	operator to accept something other than the invoking 
	object as the first operand with the invoking object
	as the second. 
	} 
\end{minipage} 
\vfill

\noindent 3. 
\begin{minipage}[t]{11.5 cm}
	Does a nonmember function have to be a friend function
	to access a class's members?
\end{minipage} \\[1ex]
\phantom{3. } 
\begin{minipage}[t]{11.5 cm}
	{\slshape 
	If the nonmember function is to access the class's members
	directly, then yes, it must be a friend function.
	If the nonmember function is to access the class's members
	indirectly (by invoking a member function, for example)
	then no, it does not need to be a friend function.
	} 
\end{minipage} 
\vfill

\noindent 4. 
\begin{minipage}[t]{11.5 cm}
	Use a friend function to overload the multiplication operator for the \texttt{Stonewt} class; have the operator multiply the \texttt{double} value by the \texttt{Stone} value.
\end{minipage} \\[1ex]
\phantom{2. } 
\begin{minipage}[t]{11.5 cm}
	{\slshape 
	See the following code:
	}
	\begin{verbatim}
	// function prototype
	friend Stonewt operator*(double x, const Stonewt & s);

	// definition
	Stonewt operator*(double x, const Stonewt & s)
	{
	    return s*x;
	}
	\end{verbatim}
\end{minipage} 
\vfill
\newpage

\phantom{\quad} 
\vfill
\noindent 5. 
\begin{minipage}[t]{11.5 cm}
	Which operators cannot be overloaded?
\end{minipage} \\[1ex]
\phantom{3. } 
\begin{minipage}[t]{11.5 cm}
	{\slshape 
	We may not overload the following operators: \\
	\verb+sizeof+ \\
	\verb+.+ \\
	\verb+.*+ \\
	\verb+::+ \\
	\verb+?:+ \\
	\verb+typeid+\\
	\verb+const_cast+\\
	\verb+dynamic_cast+\\
	\verb+reinterpret_cast+\\ 
	\verb+static_cast+
	} 
\end{minipage} 
\vfill

\noindent 6. 
\begin{minipage}[t]{11.5 cm}
	What restriction applies to overloading the following operators? \texttt{=, (), [],} and \texttt{->}.
\end{minipage} \\[1ex]
\phantom{3. } 
\begin{minipage}[t]{11.5 cm}
	{\slshape 
	We may only use member functions to overload those four
	operators.
	}
\end{minipage} 
\vfill

\noindent 7. 
\begin{minipage}[t]{11.5 cm}
	Define a conversion function for the \texttt{Vector} class that converts a \texttt{Vector} object to a type \texttt{double} value that represents the vector's magnitude.
\end{minipage} \\[1ex]
\phantom{2. } 
\begin{minipage}[t]{11.5 cm}
	{\slshape 
	See the following code:
	}
	\begin{verbatim}
	// prototype
	operator double() const;

	// definition
	Vector::operator double() const
	{
	    return mag;
	}
	\end{verbatim}
\end{minipage} 
\vfill


\end{document}

Here is the format for questions that include multiple parts:

\noindent X. 
\begin{minipage}[t]{11.5 cm}
	The question
\end{minipage} \\[1ex]
\phantom{1. }a.
\begin{minipage}[t]{11.5 cm}
	part a \\[1ex]
	{\slshape The answer.} \\
	{} % space for comments to be included
\end{minipage} \\[1ex]
\phantom{1. }b.
\begin{minipage}[t]{11.5 cm}
	part b \\[1ex]
	{\slshape The answer}\\
	{} % space for comments to be included
\end{minipage} \\[1ex]
\phantom{1. }c.
\begin{minipage}[t]{11.5 cm}
	part c \\[1ex]
	{\slshape The answer} \\
	{} % space for comments to be included
\end{minipage} \\[1ex]
\phantom{1. }d.
\begin{minipage}[t]{11.5 cm}
	part d \\[1ex]
	{\slshape The answer} \\
	{} % space for comments to be included
\end{minipage}
\vfill

\noindent X. 
\begin{minipage}[t]{11.5 cm}
	The question
\end{minipage} \\[1ex]
\phantom{3. } 
\begin{minipage}[t]{11.5 cm}
	{\slshape The answer.} 
\end{minipage} 
\vfill
