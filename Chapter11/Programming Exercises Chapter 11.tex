\documentclass[10 pt]{amsart}

\usepackage{amssymb,latexsym}
\usepackage{graphicx, setspace}
% the setspace package allows us use of the 
% spacing environment (\begin{spacing}{second arg}) where
% second arg is a number to multiply to the spacing factor.
% use 2 for double space, 1 for single space, etc.

% For the cpart environment, although it would probably be better in the
% future to implement this with a list environment.
\newlength{\cgap}
\settowidth{\cgap}{\textbf{x. }}
\newlength{\cwidth}
\setlength{\cwidth}{\textwidth}
\addtolength{\cwidth}{-\cgap}
\newenvironment{cpart}[2][\cwidth]
	{%		
		\\ %
		\textbf{#2. }%
		\begin{minipage}[t]{#1}%
		\setlength{\parindent}{0pt}%
		\setlength{\parskip}{2ex}%
	}
	{%
		\end{minipage}%
	}
\newenvironment{cpartContinued}[2][\cwidth]
	{%		
		\\ %
		\textbf{#2. (continued)}%
		\\
		\phantom{#2. }
		\begin{minipage}[t]{#1}%
		\setlength{\parindent}{0pt}%
		\setlength{\parskip}{2ex}%
	}
	{%
		\end{minipage}%
	}


% set paragraph spacing like that in the book
\setlength{\parindent}{0pt}
\setlength{\parskip}{2ex}

% these new commands will make typing different formats easier.
\newcommand{\ttt}[1]{\texttt{#1}}
\newcommand{\ttb}[1]{\pmb{\texttt{#1}}}
\newcommand{\tbs}{\textbackslash}
% Macros, all must be filled out
\newcommand{\ChapNum}{11}

\begin{document}
	\title
	[Chapter \ChapNum]
	{C++ Primer Plus, 5$^\text{th}$ Edition \\
	Programming Exercises \\
	Chapter \ChapNum}

	\maketitle

	\begin{cpart}{1}
		Modify Listing 11.15 so that it writes the successive locations
		of the random walker in a file.
		Label each position with the step number.
		Also have the program write the initial conditions (target
		distance and step size) and the summarized results to the
		file.
		The file contents might look like this:

		{\ttfamily
			Target Distance: 100, Step Size: 20 \\
			0: (x,y) = (0, 0) \\
			1: (x,y) = (-11.4715, 16.383) \\
			2: (x,y) = (-8.68807, -3.42232) \\
			\ldots \\
			26: (x,y) = (42.2919, -78.2594) \\
			27: (x,y) = (58.6429, -89.7309) \\
			After 27 steps, the subject has the following location: \\
			(x,y) = (58.6749, 089.7309) \\
			or
			(m,a) = (107.212, -56.8194) \\
			Average outward distance per step = 3.79081
		}
	\end{cpart}

	\begin{cpart}{2}
		Modify the \ttt{Vector} class header and implementation files
		(Listings 11.13 and 11.14) so that the magnitude and angle are
		no longer stored as data components.
		Instead, they should be calculated on demand when the
		\ttt{magval()} and \ttt{angval()} methods are called.
		You should leave the public interface unchanged (the same 
		public methods with the same arguments), but alter the private
		section, including some of the private method, and the 
		method implementations.
		Test the modified version with Listing 11.15, which
		should be left unchanged because the public interface of the
		\ttt{Vector} class is unchanged.
	\end{cpart}

	\begin{cpart}{3}
		Modify Listing 11.15 so that instead of reporting the results
		of a single trial for a particular target/step combination, 
		it reports the highest, lowest, and average number of steps
		for $N$ trials, where $N$ is an integer entered by the user.
	\end{cpart}

	\begin{cpart}{4}
		Rewrite the final \ttt{Time} class example (Listings 11.10, 
		11.11, and 11.12) so that all the overloaded operators are
		implemented using friend functions.
	\end{cpart}

	\begin{cpart}{5}
		Rewrite the \ttt{Stonewt} class (Listings 11.16 and 11.17)
		so that it has a state member that governs whether the object
		interpreted in stone form, integer pounds form, or floating-point
		form. 
		Overloaded the \ttt{<<} operator to replace the \ttt{show\_stn()}
		and \ttt{show\_lbs()} methods.
		Overload the addition, subtraction, and multiplication 
		operators so that one can add, subtract, and multiply 
		\ttt{Stonewt} values.
		Test your class with a short program that uses all the class
		methods and friends.
	\end{cpart}

	\begin{cpart}{6}
		Rewrite the \ttt{Stonewt} class (Listing 11.16 and 11.17)
		so that it overloads all six relational operators.
		The operators should compare the \ttt{pounds} member and 
		return a type \ttt{bool} value.
		Write a program that declares an array of six \ttt{Stonewt}
		objects and initializes the first three objects in the
		array declaration.
		Then it should use a loop to read in values used to set the
		remaining three array elements.
		Then it should report the smallest element, the largest
		element, and how many elements are greater or equal to
		11 stone.
		(The simplest approach is to create a \ttt{Stonewt} object
		initialized to 11 stone and to compare the other objects with
		that object.)
	\end{cpart}

	\begin{cpart}{7}
		A complex number has two parts: a real part and an imaginary
		part. 
		One way to write an imaginary number is this: (3.0, 4.0).
		Here 3.0 is the real part and 4.0 is the imaginary part.
		Suppose a = (A, Bi) and c = (C, Di).
		Here are some complex operations: 
		\begin{itemize}
			\item Addition: a + c = (A + C, (B + D)i)
			\item Subtraction: a - c = (A - C, (B - D)i) 
			\item Multiplication: 
				a $\times$ c = (A $\times$ C - B $\times$ D, 
									(A $\times$ D + B $\times$ C)i)
			\item Multiplication: (x is a real number): 
						x $\times$ c = (x $\times$ C, x $\times$ Di)
			\item Conjugation: $\sim$a = (A, -Bi)
		\end{itemize}

		Define a complex class so that the following program can
		use it with correct results.
		{\ttfamily
			\begin{tabbing}
				\phantom{\qquad}\=\phantom{\qquad}\=\phantom{\qquad}\= \\
				\#include <iostream> \\
				using namespace std; \\
				\#include "complex0.h"  // to avoid confusion with complex.h \\
				int main() \\
				\{
				\+ \\
				complex a(3.0, 4.0);  // initialize to (3,4i) \\
				complex c; \\
				cout << "Enter a complex number (q to quit):
							\textbackslash n"; \\
				while (cin >> c)
				\{ 
				\+ \\
				cout << "c is " << c << '\textbackslash n'; \\
				cout << "complex conjugate is " << $\sim$c << '\tbs n'; \\
				cout << "a is " << a << '\tbs n'; \\
				cout << "a + c is " << a + c << '\tbs n'; \\
				cout << "a - c is " << a - c << '\tbs n'; \\
				cout << "a * c is " << a * c << '\tbs n'; \\
				cout << "2 * c is " << 2 * c << '\tbs n'; \\
				cout << "Enter a complex number (q to quit): '\tbs n';
				\- \\
				\} \\
				cout << "Done!\tbs n"; \\
				return 0;
				\- \\
				\}
			\end{tabbing}
		}		
		Note that you have to overload the \ttt{<<} and \ttt{>>}
		operators.
		Many systems already have complex support in a \ttt{complex.h}
		header file, so use \ttt{complex0.h} to avoid conflicts.
		Use \ttt{const} whenver warranted.
	\end{cpart}
	\newpage

	\begin{cpartContinued}{7}
		Here is a sample run of the program:
		
		{\ttfamily
			Enter a complex number (q to quit): \\
			real:\ttb{ 10} \\
			imaginary:\ttb{ 12} \\
			c is (10,12i) \\
			complex conjugate is (10,-12i) \\
			a is (3,4i) \\
			a + c is (13, 16i) \\
			a - c is (-7, 8i) \\
			a * c is (-18, 76i) \\
			2 * c is (20, 24i) \\
			Enter a complex number (q to quit): \\
			real:\ttb{ q} \\
			Done!
		}

		Note that \ttt{cin >> c}, through overloading, now 
		prompts for real and imaginary parts.
	\end{cpartContinued}

\end{document}

regarding tabbing environments:
\= (set tab)
\> (advance to next tab stop)
\<
\+ (indent; move margin right)
\- (unindent; move margin left)
\'
\`
\\ (end of line; newline)
\kill (ignore preceding text; use only for spacing)

use \hspace{...} if you prefer

		{\ttfamily
			\begin{tabbing}
				\phantom{\qquad}\=\phantom{\qquad}\=\phantom{\qquad}\= \\
		
			\end{tabbing}
		}











