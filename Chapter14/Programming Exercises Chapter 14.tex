\documentclass[10 pt]{amsart}

\usepackage{amssymb,latexsym}
\usepackage{graphicx, setspace}
% the setspace package allows us use of the 
% spacing environment (\begin{spacing}{second arg}) where
% second arg is a number to multiply to the spacing factor.
% use 2 for double space, 1 for single space, etc.

% For the cpart environment, although it would probably be better in the
% future to implement this with a list environment.
\newlength{\cgap}
\settowidth{\cgap}{\textbf{x. }}
\newlength{\cwidth}
\setlength{\cwidth}{\textwidth}
\addtolength{\cwidth}{-\cgap}
\newenvironment{cpart}[2][\cwidth]
	{%		
		\\ %
		\textbf{#2. }%
		\begin{minipage}[t]{#1}%
		\setlength{\parindent}{0pt}%
		\setlength{\parskip}{2ex}%
	}
	{%
		\end{minipage}%
	}
\newenvironment{cpartContinued}[2][\cwidth]
	{%		
		\\ %
		\textbf{#2. (continued)}%
		\\
		\phantom{#2. }
		\begin{minipage}[t]{#1}%
		\setlength{\parindent}{0pt}%
		\setlength{\parskip}{2ex}%
	}
	{%
		\end{minipage}%
	}


% set paragraph spacing like that in the book
\setlength{\parindent}{0pt}
\setlength{\parskip}{2ex}

% these new commands will make typing different formats easier.
\newcommand{\ttt}[1]{\texttt{#1}}
\newcommand{\ttb}[1]{\pmb{\texttt{#1}}}
\newcommand{\tbs}{\textbackslash}
% Macros, all must be filled out
\newcommand{\ChapNum}{14}

\begin{document}
	\title
	[Chapter \ChapNum]
	{C++ Primer Plus, 5$^\text{th}$ Edition \\
	Programming Exercises \\
	Chapter \ChapNum}

	\maketitle

	\begin{cpart}{1}
		The \ttt{Wine} class has a \ttt{string} class object member 
		(see Chapter 4) that holds the name of a wine and a 
		\ttt{Pair} object (as discussed in this chapter) of 
		\ttt{valarray<int>} objects (as discussed in this chapter).
		The first member of each \ttt{Pair} object holds the 
		vintage years, and the second ember holds the number of
		bottles owned for the corresponding particular vintage
		year.
		For example, the first \ttt{valarray} object of the \ttt{Pair}
		object might hold the years 1988, 1992, 1996, and the second
		\ttt{valarray} object might hold the bottle counts 24, 48, 
		and 144. 
		It may be convenient for \ttt{Wine} to have an \ttt{int}
		member that stores the number of years. 
		Also, some \ttt{typedef}s might be useful to simplify the
		coding. 

		{\ttfamily
			typedef std::valarray<int> ArrayInt; \\
			typedef Pair<ArrayInt, ArrayInt> PairArray;
		}

		Thus the \ttt{PairArray} type represents the type
		\ttt{Pair<std::valarray<int>, std::valarray<int> >}.
		Implement the \ttt{Wine} class by using containment.
		The class hould have a default constructor and
		at least the following constructors: 

		{ \ttfamily
			// initialize label to 1, number of years to y,  \\
			// vintage years to yr[], bottles to bot[] \\
			Wine(const char * l, int y, const int yr[], const int bot[]); \\
			// initialize label to l, number of yeras to y, \\
			// create array objects of length y \\
			Wine(const char * l, int y); 
		}

		The \ttt{Wine} class should have a method \ttt{GetBottles()}
		that, given a \ttt{Wine} object with \ttt{y} years, 
		prompts the user to enter the corresponding number of vintage
		years and bottle counts.
		A method \ttt{Label()} should return a reference to the wine
		name. 
		A method \ttt{sum()} should return the total number of
		bottles in the second \ttt{valarray<int>} object in the
		\ttt{Pair} object.

		The program should prompt the user to enter a wine name,
		the number of elements of the array, and the year and bottle
		count information for each array element. 
		The program should use this data to construct a \ttt{Wine}
		object and then display the information stored in the
		object.
	\end{cpart}
	\newpage
	\begin{cpartContinued}{1}
		For guidance, here's a sample test program:
		{\ttfamily
			\begin{tabbing}
				\phantom{\qquad}\=\hspace{5cm}\=\hspace{4cm}\= \\
				// pe14-1.cpp -- using Wine class with containment \\
				\#include <iostream> \\
				\#include "winec.h" \\
				\\
				int main (void) \\
				\{
				\+ \\
					using std::cin; \\
					using std::cout; \\
					using std::endl; \\
					\\
					cout << "Enter name of wine: "; \\
					char lab[50]; \\
					cin.getline(lab, 50); \\
					cout << "Enter number of years: "; \\
					int yrs; \\
					cin >> yrs; \\
					\\
					Wine holding(lab, yrs); \> 
						// store label, year, give arrays yrs elements \\
					holding.GetBottles(); \>
						// solicit input for year, bottle count \\
					holding.Show(); \> 
						// display object contents \\
					\\
					const int YRS = 3; \\
					int y[YRS] = {1993, 1995, 1998}; \\
					int b[YRS] = {48, 60, 72}; \\
					// create new object, initialize using data in arrays 
						y and b \\
					Wine more("Gushing Grape Red", YRS, y, b); \\
					more.Show(); \\
					cout << "Total bottles for " << more.Label() \> \> 
						// use Label() method \\
					\phantom{cout} << ": " << more.sum() << endl; \> \> 
						// use sum() method \\
					cout << "Bye\tbs n"; \\
					return 0; \\
				\< \}
			\end{tabbing}
		}
	\end{cpartContinued} 
	\newpage
	\begin{cpartContinued}{1}
		And here's some sample output:

		{\ttfamily
			\begin{tabbing}
				\hspace{1.8 cm}\=\hspace{2cm}\=\hspace{4cm}\= \\
				Enter name of wine:\ttb{ Gully Wash} \\
				Enter number of years:\ttb{ 4} \\
				Enter Gully WAsh data for 4 year(s): \\
				Enter year:\ttb{ 1988} \\
				Enter bottles for that year:\ttb{ 42} \\
				Enter year:\ttb{ 1994} \\
				Enter bottles for that year:\ttb{ 58} \\
				Enter year:\ttb{ 1998} \\
				Enter bottles for that year:\ttb{ 122} \\
				Enter year:\ttb{ 2001} \\
				Enter bottles for that year:\ttb{ 144} \\
				Wine: Gully Wash \\
					\> Year \> Bottles \\
					\> 1988 \> 42 \\
					\> 1994 \> 58 \\
					\> 1998 \> 122 \\
					\> 2001 \> 144 \\
				Wine: Gusing Grape Red \\
					\> Year \> Bottles \\
					\> 1993 \> 48 \\
					\> 1995 \> 60 \\
					\> 1998 \> 72 \\
				Total bottles for Gushing Grape Red: 180 \\
				Bye
			\end{tabbing}
		}
	\end{cpartContinued}

	\begin{cpart}{2}
		This exercise is the same as Programming Exercise 1, except that
		you should use private inheritance instead of containment.
		Again, a few \ttt{typedef}s might prove handy.
		Also, you might contemplate the meaning of statements
		such as the following: 

		{\ttfamily 
			PairArray::operator=(PairArray(ArrayInt(), ArrayInt())); \\
			cout << (const string \&)(*this);
		}

		The class should work with the same test program as shown in
		Programming Exercise 1.
	\end{cpart}

	\begin{cpart}{3}
		Define a \ttt{QueueTp} template.
		Test it by creating a queue of pointers-to-\ttt{Worker}
		(as defined in Listing 14.10) and using the queue in a program
		similar to that in Listing 14.12.
	\end{cpart}

	\begin{cpart}{4}
		A \ttt{Person} class holds the first name and the last name of
		a person.
		In addition to its constructors, it has a \ttt{Show()} method
		that displays both names.
		A \ttt{Gunslinger} class derives virtually from the \ttt{Person}
		class.
		It has a \ttt{Draw()} member that returns a type \ttt{double}
		value representing a gunslinger's draw time.
		The class also has an \ttt{int} member representing the
		number of notches on a gunslinger's gun.
		Finally, it has a \ttt{Show()} function taht displays 	
		all this information.

		A \ttt{PokerPlayer} class derives virtually from the
		\ttt{Person} class.
		It has a \ttt{Draw()} member that returns a random number
		in the range 1 through 52, representing a card value.
		(Optionally, you could define a \ttt{Card} class with 
		a suit and face value members and use a \ttt{Card} return 
		value for \ttt{Draw()}.)
		The \ttt{PokerPlayer} class uses the \ttt{Person} \ttt{show()}
		function.
		The \ttt{BadDude} class derives publically from the
		\ttt{Gunslinger} and \ttt{PokerPlayer} classes.
		It has a \ttt{Gdraw()} member that returns a bad dude's 
		draw time and a \ttt{Cdraw()} member that returns the
		next card drawn.
		It has an appropriate \ttt{Show()} function.
		Define all these classes and methods, along with any other
		necessary methods (such as methods for setting object values)
		and test them in a simple program similar to that in Listing
		14.12.
	\end{cpart}

	\begin{cpart}{5}
		Here are some class declarations:
		{\ttfamily
			\begin{tabbing}
				\phantom{\qquad}\=\hspace{10ex}\=\hspace{8ex}\hspace{8ex}\= \\
				// emph.h -- header file for abstr\_emp class and children \\
				\\
				\#include <iostream> \\
				\#inlcude <string> \\
				\\
				class abstr\_emp \\
				\{
				\+ \\
				\< private: \\
					std::string fname; \> \> // abstr\_emp's first name \\
					std::string lname; \> \> // abstr\_emp's last name \\
					std::string job; \\
				\< public: \\
					abstr\_emp(); \\
					abstr\_emp(const std::string \& fn, const std::string \&
						ln, \\
					\phantom{abstr\_emp(}const std::string & j); \\
					virtual void ShowAll() const;   
						// labels and shows all data \\
					virtual void SetAll();     // prompts user for values \\
					friend std::ostream \& operator<<(std::ostream \&
						os, const abstr\_emp \& e); \\
					// just displays first and last name \\
					virtual $\sim$abstr\_emp() = 0;    // virtual base class \\
				\< \}; \\
				\- \\
				class employee : public abstr\_emp \\
				\{
				\+ \\
				\< public: \\
					employee(); \\
					employee(const std::string \& fn, const std::string
						\& ln, \\
					\phantom{employee(}const std::string & j); \\
					virtual void ShowAll() const; \\
					virtual void SetAll(); \\
				\< \}; \\
				\- \\
				class manager: virtual public abstr\_emp \\
				\{
				\+ \\
				\< private: \\
					int inchargeof; \hspace{10ex} 
						// number of abstr\_emps managed \\
				\< protedted: \\
					int InChargeOf() const \{ return inchargeof; \}
						//output \\
					int \& InChargeOf()\{ return inchargeof; \} 
						// input \\
				\< public: \\
					manager(); \\
					manager(const std::string \& fn, const std::string \& ln, \\
					\phantom{manager(}const std::string \& j, int ico = 0); \\
					manager(const abstr\_emp \& e, int ico); \\
					manager(const manager \& m); \\
					virtual void ShowAll() const; \\
					virtual void SetAll(); \\
				\< \}; \\
			\end{tabbing}
		}
	\end{cpart}
	\newpage
	\begin{cpartContinued}{5}
		{\ttfamily
			\begin{tabbing}
				\phantom{\qquad}\=\hspace{10ex}\=\hspace{8ex}\hspace{8ex}\= \\
				class fink: virtual public abstr\_emp \\
				\{
				\+ \\
				\< private: \\
					std::string reportsto; \hspace{8ex} // to whom fink reports \\
				\< protected: \\
					const std::string ReportsTo() const \{ return reportsto; \} \\
					std::string \& ReportsTo()\{ return reportsto; \} \\
				\< public: \\
					fink(); \\
					fink(const std::string \& fn, const std::string \& ln, \\
					\phantom{fink(}const std::string \& j, const std::string
						\& rpo); \\
					fink(const abstr\_emp \& e, const std::string \& rpo); \\
					fink(const fnk \& e); \\
					virtual void ShowAll() const; \\
					virtual void SetAll(); \\
				\< \}; \\
				\- \\
				class highfink: public manager, public fink // management fink \\
				\{
				\+ \\
				\< public: \\
					highfink(); \\
					highfink(const std::string \& fn, const std:: string \& 
						ln, \\
					\phantom{highfink(}const std::string \& j, const
						std::string \& rpo, \\
					\phantom{highfink(}int ico); \\
					highfink(const abstr\_emp \& e, const std::string \& rpo, 
						int ico); \\
					highfink(const fink \& f, int ico); \\
					highfink(const manager \& m, const std::string \& rop); \\
					highfink(const highfink \& h); \\
					virtual void ShowAll() const; \\
					virtual void SetAll(); \\
				\< \}; 
			\end{tabbing}
		}
		Note that the class heirarchy uses MI with a virtual base class,
		so keep in mind the special rules for constructor initialization
		lists for that case.
		Also note the presence of some protected-access methods.
		The simplifies the code for some of the \ttt{highfink}
		methods.
		(Note, for example, that if \ttt{highfink::ShowAll()} simply
		calls \ttt{fink::ShowAll()} and \ttt{manager::ShowAll()}, it
		winds up calling \ttt{abstr\_emp::ShowAll()} twice.)
		Provide teh class method implementations and test classes in 
		a program.
		Here is a minimal test program: 
	\end{cpartContinued}
	\newpage
	\begin{cpartContinued}{5}
		{\ttfamily
			\begin{tabbing}
				\phantom{\qquad}\=\phantom{\qquad}\=\phantom{\qquad}\= \\
				// pe14-5.cpp \\
				// useemp1.cpp -- using the abstr\_emp classes \\
				\\
				\#include <iostream> \\
				using namespace std; \\
				\#include "emp.h" \\
				\\
				int main(void) \\
				\{
				\+ \\
					employee em("Trip", "Harris", "Thumper"); \\
					cout << em << endl; \\
					em.ShowAll(); \\
					\\
					manager ma("Amorphia", "Spindragon", "Nuancer", 5); \\
					cout << ma << endl; \\
					ma.ShowAll(); \\
					\\
					fink fi("Matt", "Oggs", "Oiler", "Juno Barr"); \\
					cout << fi << endl; \\
					fi.ShowAll(); \\
					highfink hf(ma, "Curly Kew"); // recruitment? \\
					hf.ShowAll(); \\
					cout << "Press a key for next phrase:\tbs n"; \\
					cin.get(); \\
					highfink hf2; \\
					hf2.SetAll(); \\
					\\
					cout << "Using an abstr\_emp * pointer:\tbs n"; \\
					abstr\_emp * tri[4] = {\&em, \&fi, \&hf, \&hf2}; \\
					for (int i = 0; i < 4; i++) \\
					\> tri[i]->ShowAll(); \\
					\\
					return 0; \\
				\< \}
			\end{tabbing}
		}		

		Why is no assignment operator define? 

		Why are \ttt{ShowAll()} and \ttt{SetAll()} virtual?

		Why is \ttt{abstr\_emp} a virtual base class? 

		Why does the \ttt{highfink} class have no data section? 

		Why is only one version of the \ttt{operator<<()} needed?

		What would happen if the end of the program were replaced
			this code?

		{\ttfamily
			abstr\_emp tri[4] = \{em, fi, hf, hf2\}; \\
			for(int i = 0; i < 4; i++) \\
			\phantom{for(}tri[i].ShowAll();
		}

	\end{cpartContinued}

\end{document}

regarding tabbing environments:
\= (set tab)
\> (advance to next tab stop)
\<
\+ (indent; move margin right)
\- (unindent; move margin left)
\'
\`
\\ (end of line; newline)
\kill (ignore preceding text; use only for spacing)

use \hspace{...} if you prefer

		{\ttfamily
			\begin{tabbing}
				\phantom{\qquad}\=\phantom{\qquad}\=\phantom{\qquad}\= \\
		
			\end{tabbing}
		}











